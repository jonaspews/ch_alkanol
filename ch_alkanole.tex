%% Dieser Quelltext ist in der Kodierung UTF-8 zu speichern und mit lualatex
%% zu kompilieren.

% !TeX program = lualatex

\documentclass{scrartcl}  % KOMA-Scritp equivalent to \documentclass{article}
% \documentclass[12pt,a4paper]{article}
% \usepackage[utf8]{inputenc}
% TODO: fix \setdefaultlanguage problem while compiling
% \setdefaultlanguage[spelling=new, babelshorthands=true]{german}  
\usepackage{polyglossia}  % for LuaLatex
\usepackage{fontspec}  % for LuaLatex
\usepackage{luacode}  % for LuaLatex
\renewcommand{\familydefault}{\sfdefault}  % change serif (default-value) to SansSerif version of LaTex-font
\usepackage{amsmath}  % recommended as an adjunct to serious mathematical typesetting
\usepackage{amsfonts}  % extended set of fonts for use in mathematics
\usepackage{amssymb}  % allows the use of various special characters and symbols
\usepackage{graphicx}  % providing interface for optional arguments to the \includegraphics command
\usepackage{array}  % extended implementation of the array and tabular environments
\usepackage{xcolor}  % to use broader color plate
\usepackage{hyperref} % to create hypertext links in PDF (jump from content overview to according pages)
\usepackage{enumitem} % for enumerate with letters
\usepackage{textcomp} % for copyright symbol
\usepackage{tcolorbox} % for making boxes around stuff
\usepackage{wrapfig}  % to wrap images and text
\usepackage{lipsum}  % generates filler text
\usepackage{pdflscape}  % for using landscape format in between
\usepackage{smartdiagram}  % to create cooler diagrams
% TODO: TODO: generates clash with \xcolor package..fix later
% \usepackage[table]{xcolor}  % for coloring rows in tables, 
\usepackage{floatrow}  % provides many ways to customize layouts of floating environments
\usepackage{caption}  % for more formating options for captions
\usepackage{gensymb}  % to use symbols
\usepackage{siunitx}  % to use symbols and units
\usepackage[scale=1.5]{ccicons}  % for Creative Commons icons

% needed for chemistry notations
\usepackage{chemmacros}  % used for chemistry stuff
\usepackage{chemfig}  % used for chemistry stuff
\usepackage[version=4]{mhchem}   % used for chemistry stuff

\tcbuselibrary{skins}  % for textcolorbox

% page layout for worksheets
\usepackage[left=1.50cm, right=1.50cm, top=2.00cm, bottom=2.00cm]{geometry}
\usepackage[headsepline,footsepline]{scrlayer-scrpage}
%scrpage2
\pagestyle{scrheadings}

% this luacode has no function, except to make sure the document is compiled using LuaLaTex
\begin{luacode*}
	--[[
		print("this is just a comment")
	--]]
\end{luacode*}


% Information for Header and Footer
% *********************************

\ihead{\textsf{Chemie}}
\chead{\textbf{\textsf{Alkanole}}}
\ohead{\textsf{2022}}
% \ohead{90min}
% \ifoot{Fußzeile innen}
\cfoot{\textsf{\pagemark}}
%\ofoot{Fußzeile außen}


\author{Jonas Pews}
\title{Alkanole - Ein Skript zum selbstständigen Lernen}
\date{\today}

% Some Info about this document
% *****************************
%

% TODO: write Text how this doc can be used in school. also about RLP Brandenburg/Berlin

% TODO: write more comments to explain/comment :-) the Tex-Code 

\begin{document}

% ***** start copy here *********************************************************
% *******************************************************************************
%                __ _           _   _                            _               
%      _ __ ___ / _| | ___  ___| |_(_) ___  _ __        ___  ___| |_ _   _ _ __  
%     | '__/ _ \ |_| |/ _ \/ __| __| |/ _ \| '_ \ _____/ __|/ _ \ __| | | | '_ \ 
%     | | |  __/  _| |  __/ (__| |_| | (_) | | | |_____\__ \  __/ |_| |_| | |_) |
%     |_|  \___|_| |_|\___|\___|\__|_|\___/|_| |_|     |___/\___|\__|\__,_| .__/ 
%                                                                         |_|    
%
%		only use this when you always use the same spiderweb-diagram
% 		otherwise you have to copy this next to your actual diagramm and change the options
%		at least this is my information january 2021
		
		\renewcommand{\D}{5} % number of dimensions (config option), in this cas it fits the reflection questions
		\renewcommand{\U}{6} % number of scale units (config option)
		\newdimen\R % maximal diagram radius (config option)
		\R=3.5cm 
		\newdimen\L % radius to put dimension labels (config option)
		\L=4.5cm
		\renewcommand{\A}{360/\D} % calculated angle between dimension axes 

% *******************************************************************************
% ***** end copy here ***********************************************************


	\maketitle
	
	\begin{center}
		\ccbyncsa
	\end{center}
	

\pagebreak

	% TODO: soll auf Deutsch "Inhaltsverzeichnis dastehen
	\tableofcontents

\pagebreak
	
		
		\section{Einführung}
	
			Mit Hilfe dieses Skriptes sollst du dir das Thema Alkanole selbstständig erarbeiten. Selbstständig bedeutet wirklich SELBST und STÄNDIG. Der Unterricht im klassischen Sinne hat aufgehört. Du kannst dein eigenes Tempo bestimmen und dir deine eigenen Partner oder Partnerin suchen. Sollte die Lehrkraft nicht da sein, hast du nun immer das Material um selbstständig zu arbeiten.
			Die Lehrkraft soll dir dabei als Berater zur Seite stehen. Wenn du Fragen hast oder auf Probleme stößt, die du weder allein noch im Team lösen konntest, dann fragst du nach!
			Ansonsten ist dieses Skript so konzipiert, dass du dich selbst kontrollieren kannst. Wenn du einen Aufgabenbereich abgeschlossen hast, gehst du zur Lehrkraft und lässt dir die Lösungen geben. Analysiere deine möglichen Fehler. Bedenke: "Aus Fehlern wird man klug!"
			Für jedes Kapitel dieser Reihe ist angegeben, wie du vorgehen solltest. Die Vorüberlegungen sollen dir helfen Wissen zu reaktivieren oder Wissenslücken zu schließen. Wenn du dich an die vorgegebenen Vorgehensweisen hältst, solltest du keine Probleme haben.
			Die Reihe ist für ungefähr 8 Wochen ausgelegt.
			Vielleicht fragst du dich jetzt, warum die Lehrer so faul sein dürfen und du jetzt alles allein machen musst. Der Grund ist recht einfach: die Lehrer haben alles bereits so vorbereitet, dass du dich intensiv mit einem Thema beschäftigen kannst. Dadurch bleibt es besser in deinem Wissensspeicher – auch als Gehirn bekannt – hängen. Du lernst effektiver die Inhalte, verbesserst dein eigenes Zeitmanagement und analysierst deine eigenen Fähig- und Fertigkeiten.
			Als zusätzliche Leistung musst du ein Lehrvideo erstellen. Dazu kannst du dir ein Thema aus dem Skript aussuchen. Bildet eine Gruppe aus maximal vier Leuten. Das Video sollte nicht länger als vier Minuten sein. Es wird in deine Note einfließen.
	
			\subsection{Symbole im Skript}
	
				% TODO: change size of symbols
	
				\begin{tcolorbox}[enhanced,
					colback=white,
					colframe=darkgray,
					fonttitle=\sffamily\bfseries\large, 
					title=Informationstexte,  % search keyword::Informationstexte 
					attach boxed title to top left={xshift=3.2mm,yshift=-0.50mm},
					boxed title style={skin=enhancedfirst jigsaw,size=small,arc=1mm,bottom=-1mm,colframe=darkgray,height=0.75cm},
					colbacktitle=darkgray,
					drop lifted shadow]
					\begin{wrapfigure}{L}{0.15\textwidth}  
						\centering
						\vspace{-14pt}  % to align image with first line of text
						\includegraphics[width=0.9\textwidth]{symbols/symbol_tex_content}
					\end{wrapfigure}
					
					In diesen Texten findest du Erklärungen und Hintergründe! \newline 
					Die Quellen findest du in den Fußnoten. Diese Quellen können dir auch als Quizvorbereitung helfen. Übrigens, nicht alle Quellen sind Wikipedia. Aber es ist eine nützliche – und in Chemie akzeptierte Quelle. 
					\vspace{0.7cm}  % to fill empty space in tcolorbox
				\end{tcolorbox}
	
				\begin{tcolorbox}[enhanced,
					colback=white,
					colframe=orange!60!red,
					fonttitle=\sffamily\bfseries\large, 
					title=Wiederholung,  % search keyword::Wiederholung
					attach boxed title to top left={xshift=3.2mm,yshift=-0.50mm},
					boxed title style={skin=enhancedfirst jigsaw,size=small,arc=1mm,bottom=-1mm,colframe=orange!60!red,height=0.75cm},
					colbacktitle=orange!60!red,
					% sharp corners,
					drop lifted shadow]	
					\begin{wrapfigure}{L}{0.15\textwidth}  
						\centering
						\vspace{-14pt}  % to align image with first line of text
						\includegraphics[width=0.9\textwidth]{symbols/symbol_tex_review}
					\end{wrapfigure}
					
					An diesen Stellen sollst du dein Wissen auffrischen! \newline 
					Du solltest die entsprechenden Themen schon vorher im (Chemie-)Unterricht behandelt haben. Falls nicht, arbeite deine Wissenslücken bitte selbstständig auf. 
					\vspace{1.0cm}
				\end{tcolorbox}
				
				
				\begin{tcolorbox}[enhanced,
					colback=white,
					colframe=black,
					fonttitle=\sffamily\bfseries\large, 
					title=Internet-Quelle (URL),  % search keyword::URL
					attach boxed title to top left={xshift=3.2mm,yshift=-0.50mm},
					boxed title style={skin=enhancedfirst jigsaw,size=small,arc=1mm,bottom=-1mm,colframe=black,height=0.75cm},
					colbacktitle=black,
					drop lifted shadow]
					\begin{wrapfigure}{L}{0.15\textwidth}  
						\centering
						\vspace{-14pt}  % to align image with first line of text
						\includegraphics[width=0.9\textwidth]{symbols/symbol_tex_qrcode}
					\end{wrapfigure}
					
					Manche Online-Quellen haben nicht ins Skript gepasst. Daher kannst du mit einem Handy diese QR-Codes einlesen und so die Weblinks (URLs) öffnen. 
					\vspace{1.5cm}  % to fill empty space in tcolorbox
				\end{tcolorbox}
				
				\begin{tcolorbox}[enhanced,
					colback=white,
					colframe=green!30!black,
					fonttitle=\sffamily\bfseries\large, 
					title=Durchführung,  % search keyword::Durchführung
					attach boxed title to top left={xshift=3.2mm,yshift=-0.50mm},
					boxed title style={skin=enhancedfirst jigsaw,size=small,arc=1mm,bottom=-1mm,colframe=green!50!black,height=0.75cm},
					colbacktitle=green!50!black,
					drop lifted shadow]
					\begin{wrapfigure}{L}{0.15\textwidth}  
						\centering
						\vspace{-14pt}  % to align image with first line of text
						\includegraphics[width=0.9\textwidth]{symbols/symbol_tex_method}
					\end{wrapfigure}
					
					Dieses Symbol weißt immer auf eine Durchführung für ein Experiment hin. 
					\vspace{2.3cm}  % to fill empty space in tcolorbox
				\end{tcolorbox}
				
				\begin{tcolorbox}[enhanced,
					colback=white,
					colframe=red,
					fonttitle=\sffamily\bfseries\large, 
					title=Für schnelle Schüler\_innen,  % search keyword::schnelle_Schüler
					attach boxed title to top left={xshift=3.2mm,yshift=-0.40mm},
					boxed title style={skin=enhancedfirst jigsaw,size=small,arc=1mm,bottom=-1mm,colframe=red,height=0.75cm},
					colbacktitle=red,
					drop lifted shadow]
					\begin{wrapfigure}{L}{0.15\textwidth}  
						\centering
						\vspace{-14pt}  % to align image with first line of text
						\includegraphics[width=0.8\textwidth]{symbols/symbol_tex_faststudents}
					\end{wrapfigure}
					
					Es soll keine Langeweile aufkommen. Wenn du mit Aufträgen bereits fertig bist, während deine Gruppe noch arbeitet, kannst du dich hier noch weiter in das Thema vertiefen. 
					\vspace{1.2cm}  % to fill empty space in tcolorbox
				\end{tcolorbox}
				
				\begin{tcolorbox}
					[enhanced,
					colback=white,
					colframe=black,
					fonttitle=\sffamily\bfseries\large, 
					title=Zeit,  % search keyword::Zeit
					attach boxed title to top left={xshift=3.2mm,yshift=-0.40mm},
					boxed title style={skin=enhancedfirst
						jigsaw,size=small,arc=1mm,bottom=-1mm,colframe=black,height=0.75cm},
					colbacktitle=black,
					drop lifted shadow]
					\begin{wrapfigure}{L}{0.15\textwidth}
						\centering
						\vspace{-14pt}  % to align image with first line of text
						\includegraphics[width=0.7\textwidth]{symbols/symbol_tex_time}
					\end{wrapfigure}
					
					Das Zeitsymbol soll dir zeigen, wie lange du für das jeweilige Kapitel brauchen solltest. Diese Zeitangabe dient aber nur als Orientierung. Am Ende musst du nur die Planung deiner Lehrkraft und deine eigene Zeitplanung beachten. 
					\vspace{1.0cm}  % to fill empty space in tcolorbox
				\end{tcolorbox}
				
				
				\subsection{Bewertung}
				
					Das Thema Alkanole wird uns bis zu den Weihnachtsferien beschäftigen. In dieser Zeit musst du folgende Leistungen erbringen:
					\begin{enumerate}
						\item \textbf{Ausführliches Protokoll} zu den Eigenschaften der linearen Alkanole,
						\item (Zwei) \textbf{Tests},
						\item \textbf{Tutorial Video} (4-5min) zu einem Thema aus dem Skript, nach Rücksprache mit der Lehrkraft,
						\item \textbf{Portfolio},
						\item (mündliche) \textbf{Mitarbeit}; diese ergibt sich daraus, wie viele Aufgaben du mit der Lehrerkraft besprochen und kontrolliert hast.  Du musst alle Kapitel schaffen und mit der Lehrkraft besprechen, wenn du die volle Punktzahl erhalten willst.
					\end{enumerate}
	
					\subsubsection{Die Mitarbeit}
						
						Die Mitarbeit wird mit Hilfe eines Punktesystems berechnet. Folgendes Prinzip zählt dabei:
						\begin{itemize}
							\item 1 Punkt: Das Kapitel ist größtenteils vollständig richtig in deinem Portfolio zu finden. Du hast aber noch wenige Fehler darin oder wenige Aufgaben nicht gelöst.
							\item 2 Punkte: Das Kapitel ist vollständig und komplett richtig in deinem Portfolio zu finden.
							\item 3 Punkte: Das Kapitel ist vollständig und komplett richtig in deinem Portfolio zu finden. Die Lehrkraft hat den Inhalt durchgesehen und für korrekt befunden.
							\item 4 Punkte: Das Kapitel ist vollständig und komplett richtig in deinem Portfolio zu finden. Die Lehrkraft hat den Inhalt durchgesehen und für korrekt befunden. Die Lehrkraft hat mit dir ein Testatgespräch zum Inhalt des Kapitels geführt.
						\end{itemize}
						Am Ende der Einheit werden alle Punkte zusammengerechnet. Daraus ergibt sich die Note für die (mündliche) Mitarbeit im Unterricht.
	
					\subsubsection{Das Portfolio}
					
						Das Portfolio ist ein Teil der Arbeit und Bewertung. Zum einen dient es der Sicherung und Sammlung aller Arbeitsergebnisse. Du kannst und sollst in diesem Portfolio alles sammeln, was du an Materialien und Produkten selbst erarbeitetet hast. 
						Die zweite Funktion des Portfolios ist die Darstellung deiner eigenen Entwicklung. Mit Hilfe des Portfolios belegst du deinen Lernfortschritt und reflektierst deine Arbeitsergebnisse und deine Arbeitsweise. Diese Reflexion sollte sich  auf alle Arbeitsprozesse, wie z.B. Recherchen oder Gruppenarbeiten, beziehen. Die Selbstreflexion sollte unabhängig von den Arbeitsaufträgen der Lehrkraft erfolgen.
			
					\subsubsection{Hilfreiche Fragen für die Reflexion}
			
						\begin{center}
							\smartdiagramset{
								planet size=3cm, 
								distance planet-text=0.1,
								distance planet-satellite=5.5cm,
								% /tikz/connection planet satellite/.append style={<-}
							} 
							\smartdiagram[constellation diagram]{Reflexion, {Habe ich die Zeit effektiv genutzt?}, {Habe ich alle Aufträge gelöst?}, {Habe ich alles verstanden?},  {Habe ich gut allein gearbeitet?}, {Habe ich gut in der Gruppe gearbeitet?}, {Was kann ich in der nächsten Stunde besser machen?}}
						\end{center}
						
						\noindent Für eine schnelle Reflexion kann man dieses Diagramm benutzen. Im Portfolio sollte dann aber \textbf{zusätzlich} eine ausführliche Reflexion in Textform zu finden sein. Am Ende jedes Kapitels findet man eine Kurzreflexion. Aber im Portfolio kannst du auch öfter reflektieren. 
						
						% ***** start copy here *********************************************************
						% *******************************************************************************
						%            __ _           _   _                        _                _   
						%  _ __ ___ / _| | ___  ___| |_(_) ___  _ __         ___| |__   __ _ _ __| |_ 
						% | '__/ _ \ |_| |/ _ \/ __| __| |/ _ \| '_ \ _____ / __| '_ \ / _` | '__| __|
						% | | |  __/  _| |  __/ (__| |_| | (_) | | | |_____| (__| | | | (_| | |  | |_ 
						% |_|  \___|_| |_|\___|\___|\__|_|\___/|_| |_|      \___|_| |_|\__,_|_|   \__|
						%                                                                            
						%
						
					%	\vspace{0.3cm}
						\begin{center}
							\begin{tcolorbox}[enhanced,
								width=0.75\textwidth,
								colback=white,
								colframe=darkgray,
								fonttitle=\sffamily\bfseries\large, 
								title=Kurzreflexion,  % search keyword::Informationstexte 
								attach boxed title to top left={xshift=3.2mm,yshift=-0.50mm},
								boxed title style={skin=enhancedfirst jigsaw,size=small,arc=1mm,bottom=-1mm,colframe=darkgray,height=0.75cm},
								colbacktitle=darkgray,
								drop lifted shadow]
								
								% \begin{figure}[htbp]  % not good in ecolorbox
								\textbf{Auftrag: Bewerte deine Arbeit in der letzten Einheit selbst.} \newline
								\begin{center}
								\begin{tikzpicture}[scale=0.8]
									\path (0:0cm) coordinate (O); % define coordinate for origin
									
									% draw the spiderweb
									\foreach \X in {1,...,\D}{
										\draw (\X*\A:0) -- (\X*\A:\R);
									}
									
									\foreach \Y in {0,...,\U}{
										\foreach \X in {1,...,\D}{
									\path (\X*\A:\Y*\R/\U) coordinate (D\X-\Y);
									\fill (D\X-\Y) circle (1pt);
										}
										\draw [opacity=0.3] (0:\Y*\R/\U) \foreach \X in {1,...,\D}{
											-- (\X*\A:\Y*\R/\U)
										} -- cycle;
									}
									
									% define labels for each dimension axis (names config option)
									\path (1*\A:\L) node (L1) {\tiny Verständnis};
									\path (2*\A:\L) node (L2) {\tiny alle Aufgaben gelöst};
									\path (3*\A:\L) node (L3) {\tiny Zeit effektiv genutz};
									\path (4*\A:\L) node (L4) {\tiny Einzelarbeit};
									\path (5*\A:\L) node (L5) {\tiny Gruppenarbeit};
								
								\end{tikzpicture}
								% \caption{Diagramm Slebstreflexion}
								% \caption{Spiderweb Diagram (\D~Dimensions, \U-Notch Scale, 3 Samples)}
								% \label{fig:spiderweb}
								% \end{figure} 
								\end{center}
								\textbf{{\Large Was kannst du in der nächsten Stunde verbessern?}}
							
							\end{tcolorbox}
						\end{center}
						% *******************************************************************************
						% ***** end copy here ***********************************************************

			
		\newpage
			
						\begin{landscape}
						
						\subsubsection{Bewertungsraster für das Portfolio}
						Die Bewertung des Portfolios erfolgt zum Ende des Halbjahres und mit Hilfe dieses Bewertungsrasters. \newline
						
						\begin{tabular}{|l|*{4}{p{4.5cm}|}}  % da alle Spalten gleich sein sollen: Sternoperator {*{Anzahl n}{Spaltentyp}}
							\hline
							% *** 1. Zeile **************************************************
							\textbf{Kriterium} &
							\textbf{1BE} &
							\textbf{2BE} &
							\textbf{3BE} &
							\textbf{4BE} \\
							\hline
							% *** 2. Zeile **************************************************
							\multicolumn{5}{c}{\textbf{Formale Kriterien (Gewichtung 1)}} \\
							\hline
							% *** 3. Zeile **************************************************
							\textbf{Deckblatt} &
							In Ansätzen vorhanden &
							Vorhanden &
							Vorhanden und sorgfältig gestaltet &
							Vorhanden und sorgfältig bzw. kreativ gestaltet \\
							\hline
							% *** 4. Zeile **************************************************
							\textbf{Inhaltsverzeichnis} &
							In Ansätzen vorhanden &
							Vorhanden &
							Vorhanden, sauber und sorgfältig gestaltet &
							Vorhanden, sauber, sorgfältig bzw. kreativ gestaltet \\
							\hline
							% *** 5. Zeile **************************************************
							\textbf{Form und Sprache} &
							Ausführung ist in Teilen nicht akzeptabel &
							Ausführung ist akzeptabel &
							Ausführung ist  sauber und ordentlich &
							Ausführung ist sehr sauber, leserlich und ordentlich \\
							\hline
%							% *** 6. Zeile **************************************************
%							\textbf{Sprache} &
%							Sprachliche Defizite &
%							Zum Teil sprachliche Fehler &
%							Kaum sprachliche Fehler &
%							Angemessene fehlerfreie Sprache \\
%							\hline
							% *** 7. Zeile **************************************************
							\multicolumn{5}{c}{\textbf{Inhaltliche Kriterien (Gewichtung 2)}} \\
							\hline
							% *** 8. Zeile **************************************************
							\textbf{Dokumentation} &
							Weniger als zur Hälfte erfüllt &
							Mehr als zur Hälfte erfüllt &
							Weitgehend erfüllt &
							Vollständig erfüllt \\
							\hline
							% *** 9. Zeile **************************************************
							\multicolumn{5}{c}{\textbf{Reflexion der Arbeit und des Erkenntnisgewinns (Gewichtung 3)}} \\
							\hline
							% *** 10. Zeile **************************************************
							\textbf{Reflexion} &
							\textbf{Kaum Reflexionsfähigkeit erkennbar.} Die Kurzreflexionen wurden selten genutzt oder das Semester wurde abschließend reflektiert oder die Reflexion wurde während des Semsters manchmal vorgenommen. &
							\textbf{Reflexionsfähigkeit zum Teil erkennbar.} Die Kurzreflexionen wurden selten genutzt. Das Semester wurde abschließend reflektiert oder die Reflexion wurde während des Semsters manchmal vorgenommen. &
							\textbf{Gute Reflexionsfähigkeit erkennbar.} Die Reflexion wurde mehrfach während des Semsters vorgenommen. Die Kurzreflexionen wurden genutzt. Das Semester wurde abschließend ergänzend reflektiert.  &
							\textbf{Sehr gute Reflexionsfähigkeit erkennbar.} Die Reflexion wurde mehrfach während des Semsters vorgenommen. Die Kurzreflexionen wurden sinnvoll genutzt. Das Semester wurde abschließend ausführlich ergänzend und glaubhaft reflektiert. \\
							\hline
						\end{tabular} \newline
						
						\vspace{1cm}
						
						\noindent Die \textit{Dokumentation} der Arbeit enthält z.B. die Lösungen zu den Arbeitsaufträgen, weitere Mitschriften, Quellen, Recherchen, gezeichnete oder ausgedruckte Bilder, Mind-Maps etc. \newline
						Die durchgängige \textit{Reflexion} beinhaltet die Arbeit in der Klasse, in der Gruppe, Einzelarbeit, die Reflexion des Erkenntnisstands etc. \newline
						Die Lehrkraft kann in den Kategorien \textit{Dokumentation} und \textit{Reflexion} jeweils 1 bzw. 2BE von der Bewertung individuell abziehen.
						
						\end{landscape}
			
			%TODO: Text zur Bewertung und genauen Arbeitsweise einfügen!
			
\newpage
	\part{Fundamentum}
		\section{Die Herstellung von Ethanol}

			\textit{Herzlichen Glückwunsch! Du hast es dir in der Zombie-Apokalypse gemütlich gemacht! Du hast dein Auto zum Laufen bekommen und verstanden, wozu man organische Chemie und Erdöl braucht. Nun ist aber ein tötliches Virus aufgetaucht, dass sich von Mensch zu Mensch verbreitet. Zeit, das Desinfektionsmittel rauszuholen oder herzustellen.} \newline
			
			\begin{minipage}{0.7\textwidth}
				\noindent \textbf{Am Ende dieses Kapitels sollst du ... :}
				\begin{enumerate}
					\item ... beschreiben können, wie die natürliche herstellung von Ethanol funktioniert.
					\item ... die Reaktionsgleichung für die alkoholische Gärung aufstellen können.
					\item ... die Funktionsweise eines (Bio-)katalysators erklären können.
				\end{enumerate}
				\textbf{Vorgehensweise:}
				\begin{enumerate}
					\item Arbeite in Gruppen von 4-5 SuS.
					\item Führt das Experiment durch.
					\item Bearbeite anschließend die restlichen Arbeitsaufträge.
					\item Nachdem du den Arbeitsabschnitt erledigt hast, tausche dich mit einigen MitschülerInnen aus. 
				\end{enumerate}
				
			\end{minipage}
			\hspace{0.1\textwidth}
			\begin{minipage}{0.2\textwidth}
				\begin{tcolorbox}
					[enhanced,
					width=0.9\textwidth,
					colback=white,
					colframe=black,
					fonttitle=\sffamily\bfseries\large, 
					title=Zeit,  % search keyword::Zeit
					attach boxed title to top center={xshift=-0.0mm,yshift=-0.50mm},
					boxed title style={skin=enhancedfirst jigsaw,size=small,arc=1mm,bottom=-1mm,colframe=black,height=0.75cm},
					colbacktitle=black,
					drop lifted shadow]
					\centering
					\includegraphics[width=0.9\textwidth]{symbols/symbol_tex_time}
					
					\begin{center}
						\textbf{180min}
					\end{center}
				\end{tcolorbox}
			\end{minipage}
			
			\begin{center}
				\noindent\rule{18cm}{0.1pt}
			\end{center}


			
\newpage
			\subsection{Die alkoholische Gärung}
			
				\textbf{Auftrag: Plane ein Experiment, bei dem du Ethanol (dt. Trinkalkohol) herstellst.}
				\begin{enumerate}
					\item \textbf{Recherchiere}, wie man Ethanol mit Hilfe der alkoholischen Gärung herstellen kann.
					\item Erstelle eine Durchführung und Materialliste (Geräte und Chemikalien) für das Experiment. Begründe deine Planung und Materialliste! 
					\item \textbf{Visualisiere} deine Durchführung! Hilfe und Inspiration findest du (notfalls) auf dem Lehrertisch.
					\item Führe das \textbf{Experiment} durch und schreibe ein kurzes \textbf{Protokoll}! Kennzeichne deinen Versuch, da er einige Zeit stehen bleiben muss.
				\end{enumerate}

			\subsection{Gärung und Hefe}	
			
				\textit{Für die Herstellung von (Trink-)Alkohol bzw. Ethanol bruacht man unbedingt Hefe. Sie ist ein sehr gutes Beispiel für einen (Bio-)Katalysator. Aber wie funktioniert das nun genau?}\newline	
	
				\noindent \textbf{Auftrag: Erkläre die Funktion der Hefe beim Gärprozess.}
				\begin{enumerate}
					\item \textbf{Lies} den Text.
					\item \textbf{Erkläre} den biologischen Hintergrund der Hefe.
					\item \textbf{Formuliere} für die chemischen Gleichungen jeweils eine Wortgleichung.
					\item \textbf{Erkläre} den Begriff \textit{anaerob}.
					\item \textbf{Erkläre}, warum man das Gärgefäß mit einem Gärröhrchen verschließt.
					\item Man kann durch die alkoholische Gärung keinen Wein herstellen, der mehr als 16\% Alkohol hat. \textbf{Erkläre} warum und \textbf{nenne} den physikalischen Vorgang, mit dem man höher konzentrierten Alkohol (z.B. Schnaps) herstellen kann.
				\end{enumerate}

				\begin{tcolorbox}[enhanced,
					colback=white,
					colframe=darkgray,
					fonttitle=\sffamily\bfseries\large, 
					title=Informationstexte,  % search keyword::Informationstexte 
					attach boxed title to top left={xshift=3.2mm,yshift=-0.50mm},
					boxed title style={skin=enhancedfirst jigsaw,size=small,arc=1mm,bottom=-1mm,colframe=darkgray,height=0.75cm},
					colbacktitle=darkgray,
					drop lifted shadow]
					\begin{wrapfigure}{L}{0.15\textwidth}  
						\centering
						\vspace{-14pt}  % to align image with first line of text
						\includegraphics[width=0.9\textwidth]{symbols/symbol_tex_content}
					\end{wrapfigure}
					
					Hefe ist ein Mikroorganismus, der eine wichtige Zutat für die Gärung ist. In der Antike wurde Hefe zunächst nicht angebaut oder kultiviert, sondern war einfach ein Teil der Natur. Die Sporen siedelten sich auf allen Arten von Früchten an und die Gärung setzte ein. Hefe ist eigentlich ein Begriff für verschiedene Formen von Pilzen. Der spezifische Mikroorganismus für die Gärung ist die Hefespezies \textit{Saccharomyces Cerevisiae}. Wie der Name schon andeutet, besteht ihr Hauptziel darin, Zucker - allgemein als Saccharose bekannt - in Energie zum Überleben umzuwandeln. Dies kann auf zwei Arten geschehen. \newline
					Wenn genügend Sauerstoff während der Verdauung vorhanden ist, läuft die chemische Reaktion grundsätzlich ab:
					\begin{center}
						\ch{C6H12O6 + 6 O2 → 6 CO2 + 6 H2O} (+Energie) 
					\end{center}
					Der Zucker für die Reaktion ist Bestandteil jeder Frucht. Bei dieser Reaktion wird auf ziemlich komplizierte Weise Energie erzeugt, was wir im Moment ignorieren können. Würde man also versuchen Wein herstellen und die Gärungsflasche offen halten, dann würde man keinen Alkohol bekommen, weil nur \ch{CO2} und \ch{H2O} entsteht. \newline
					Aus diesem Grund kopieren Brauer und Winzer die Natur. Auch hier kann es zur Gärung kommen, wenn die in der Luft befindliche Hefe die Früchte vollständig bedeckt. Auf diese Weise gibt es keinen Sauerstoff, den die Hefe bei der chemischen Reaktion verwenden könnte. Sie geht in einen anaeroben Prozess über. Die Hefe muss beginnen, Alkohol zu produzieren, wenn sie überleben will.
					\begin{center}
						\ch{C6H12O6 → 2 C2H5OH + 2 CO2} (+Energie)
					\end{center}
					Dieser Prozess ist nicht so gut für die Hefe, da nicht die gleiche Energiemenge wie im ersten Prozess erzeugt wird. Ironischerweise führt dieser Prozess schließlich zum Absterben der Hefe. Alkohol ist eine giftige Substanz, was bedeutet, dass Hefe - vergleichbar mit uns Menschen - nur bestimmte Mengen verträgt. Wenn die Alkoholkonzentration etwa 16\% erreicht (abhängig von der Art der Hefe), schaltet sich die Hefe ab. Spirituosen mit einer höheren Alkoholkonzentration werden durch physikalische Verfahren hergestellt. Für den Menschen hingegen ist der zweite Prozess, den wir als alkoholischen Gärung bezeichnen, viel wünschenswerter. 
					% \vspace{0.7cm}  % to fill empty space in tcolorbox
				\end{tcolorbox}
				
				\vspace{0.3cm}
				\begin{tcolorbox}[enhanced,
					colback=white,
					colframe=red,
					fonttitle=\sffamily\bfseries\large, 
					title=Für schnelle Schüler\_innen,  % search keyword::schnelle_Schüler
					attach boxed title to top left={xshift=3.2mm,yshift=-0.40mm},
					boxed title style={skin=enhancedfirst jigsaw,size=small,arc=1mm,bottom=-1mm,colframe=red,height=0.75cm},
					colbacktitle=red,
					drop lifted shadow]
					\begin{wrapfigure}{L}{0.15\textwidth}  
						\centering
						\vspace{-14pt}  % to align image with first line of text
						\includegraphics[width=0.8\textwidth]{symbols/symbol_tex_faststudents}
					\end{wrapfigure}
					
					Ist die alkoholische Gärung eine endotherme oder exotherme Reaktion? \textbf{Begründe} deine Antwort!
					\vspace{1.2cm}  % to fill empty space in tcolorbox
				\end{tcolorbox}
			
		\newpage  % for layout purposes, might have to be changes again 
				
				\vspace{0.3cm}
				\noindent \textit{Du hast vielleicht frisches Obst oder Fruchsaft genutzt um Alkohol herzustellen. Aber wie findet eigentlich die Hefe den Zucker, den sie verbraucht? Und wie stellt sie sicher, dass es nicht andere Chemikalien, wie z.B. Wasser, sind?}
				
				\noindent \textbf{Auftrag: Erkläre, wie die Hefe die richtige Substanz für die Herstellung von Alkohol findet.}
				
				\begin{enumerate}
					\item \textbf{Lies} den Text
					\item \textbf{Beschreibe} die Reaktion des Enzyms während der chemischen Reaktion. \textbf{Verbinde} die richtigen \textbf{Abbildungen} mit den richtigen, genannten \textbf{Teilschritten und Begriffen}.
				\end{enumerate}
			
				\begin{tcolorbox}[enhanced,
					colback=white,
					colframe=darkgray,
					fonttitle=\sffamily\bfseries\large, 
					title=Informationstexte,  % search keyword::Informationstexte 
					attach boxed title to top left={xshift=3.2mm,yshift=-0.50mm},
					boxed title style={skin=enhancedfirst jigsaw,size=small,arc=1mm,bottom=-1mm,colframe=darkgray,height=0.75cm},
					colbacktitle=darkgray,
					drop lifted shadow]
					\begin{wrapfigure}{L}{0.15\textwidth}  
						\centering
						\vspace{-14pt}  % to align image with first line of text
						\includegraphics[width=0.9\textwidth]{symbols/symbol_tex_content}
					\end{wrapfigure}
					
					Enzyme wie Hefe arbeiten nur mit bestimmten Substanzen. Sie werden als substratspezifisch bezeichnet. Ein Substrat ist eine chemische Verbindung, die nur durch bestimmte Enzyme verändert werden kann. Während der Gärung ist Zucker das Substrat. \newline
					Das Substrat wird an das katalytische Zentrum des Enzyms gebunden. Dies funktioniert wie ein Schlüssel-Schloss-Paar. Beide zusammen bilden einen Enzym-Substrat-Komplex. \newline
					Danach wirkt das Enzym als Katalysator für die chemische Reaktion, bei der das Substrat reagiert. In diesem Prozess können ein oder mehrere Produkte hergestellt werden. Diese Produkte sind noch an das Enzym gebunden. Dies nennt man den Enzym-Produkt-Komplex. \newline
					Nach der Reaktion wird das Produkt verworfen. Das Enzym hat sich nicht verändert und kann an einer anderen Reaktion arbeiten. Während der Gärung reagiert der Substratzucker zu den Produkten Alkohol und Kohlendioxid. Das Enzym Hefe ist unverändert. 
					% \vspace{0.7cm}  % to fill empty space in tcolorbox
				\end{tcolorbox}
				
				\vspace{0.3cm}
				\noindent
				\begin{minipage}[t]{0.2\textwidth}
					\includegraphics[width=\textwidth]{images/05_enzym_products}
				\end{minipage}
				\begin{minipage}[t]{0.2\textwidth}
					\includegraphics[width=\textwidth]{images/03_enzymsubstratekomplex}
				\end{minipage}
				\begin{minipage}[t]{0.2\textwidth}
					\includegraphics[width=\textwidth]{images/01_enzym_substrate}
				\end{minipage}
				\begin{minipage}[t]{0.2\textwidth}
					\includegraphics[width=\textwidth]{images/02_enzym_rightsubstrate}
				\end{minipage}
				\begin{minipage}[t]{0.2\textwidth}
					\includegraphics[width=\textwidth]{images/04_enzymproductkomplex}
				\end{minipage}
				
				\vspace{3cm}
				\noindent
				\begin{minipage}[t]{0.2\textwidth}
					{\fontfamily{qag}\selectfont  % set different font
						\begin{tcolorbox}[enhanced,
							colback=white,
							colframe=teal,
							fonttitle=\sffamily\bfseries\large, 
							% title=Valentinas Logbuch, 
							attach boxed title to top left={xshift=3.2mm,yshift=-0.50mm},
							boxed title style={skin=enhancedfirst jigsaw,size=small,arc=1mm,bottom=-1mm,colframe=teal,height=0.6cm},
							colbacktitle=teal,
							drop lifted shadow]
							
							\begin{center}
								Enzym und verschiedene Substrate
							\end{center}
						\end{tcolorbox}
					}  % end different font
				\end{minipage}
				\begin{minipage}[t]{0.2\textwidth}
					{\fontfamily{qag}\selectfont  % set different font
						\begin{tcolorbox}[enhanced,
							colback=white,
							colframe=teal,
							fonttitle=\sffamily\bfseries\large, 
							% title=Valentinas Logbuch, 
							attach boxed title to top left={xshift=3.2mm,yshift=-0.50mm},
							boxed title style={skin=enhancedfirst jigsaw,size=small,arc=1mm,bottom=-1mm,colframe=teal,height=0.6cm},
							colbacktitle=teal,
							drop lifted shadow]
							
							\begin{center}
								Enzym und richtiges Substrat
							\end{center}
						\end{tcolorbox}
					}  % end different font
				\end{minipage}
				\begin{minipage}[t]{0.2\textwidth}
					{\fontfamily{qag}\selectfont  % set different font
						\begin{tcolorbox}[enhanced,
							colback=white,
							colframe=teal,
							fonttitle=\sffamily\bfseries\large, 
							% title=Valentinas Logbuch, 
							attach boxed title to top left={xshift=3.2mm,yshift=-0.50mm},
							boxed title style={skin=enhancedfirst jigsaw,size=small,arc=1mm,bottom=-1mm,colframe=teal,height=0.6cm},
							colbacktitle=teal,
							drop lifted shadow]
							
							\begin{center}
								Enzym-Substrat-Komplex
							\end{center}
						\end{tcolorbox}
					}  % end different font
				\end{minipage}
				\begin{minipage}[t]{0.2\textwidth}
					{\fontfamily{qag}\selectfont  % set different font
						\begin{tcolorbox}[enhanced,
							colback=white,
							colframe=teal,
							fonttitle=\sffamily\bfseries\large, 
							% title=Valentinas Logbuch, 
							attach boxed title to top left={xshift=3.2mm,yshift=-0.50mm},
							boxed title style={skin=enhancedfirst jigsaw,size=small,arc=1mm,bottom=-1mm,colframe=teal,height=0.6cm},
							colbacktitle=teal,
							drop lifted shadow]
							
							\begin{center}
								Enzym-Produkt-Komplex
							\end{center}
						\end{tcolorbox}
					}  % end different font
				\end{minipage}
				\begin{minipage}[t]{0.2\textwidth}
					{\fontfamily{qag}\selectfont  % set different font
						\begin{tcolorbox}[enhanced,
							colback=white,
							colframe=teal,
							fonttitle=\sffamily\bfseries\large, 
							% title=Valentinas Logbuch, 
							attach boxed title to top left={xshift=3.2mm,yshift=-0.50mm},
							boxed title style={skin=enhancedfirst jigsaw,size=small,arc=1mm,bottom=-1mm,colframe=teal,height=0.6cm},
							colbacktitle=teal,
							drop lifted shadow]
							
							\begin{center}
								Enzym und verschiedene Produkte
							\end{center}
						\end{tcolorbox}
					}  % end different font
				\end{minipage}
		\newpage  % for layout purposes, might have to be changes again		
			%	\vspace{0.3cm}
				\noindent \textit{Hefe ist nur ein Beispiel für eine riesige Liste von Katalysatoren. Im folgenden Text erfährst du, was ein Katalysator eigentlich ist und wie er funktioniert.} \newline
				
				\noindent \textbf{Auftrag: Erkläre was ein Katalysator ist und wie er funktioniert.}
				\begin{enumerate}
					\item \textbf{Lies} den Text.
					\item \textbf{Formuliere} eine Definition für den Begriff \textit{Katalysator}. \textbf{Nenne} die drei wichtigen Eigenschaften, die ein Katalysator haben muss.
					\item \textbf{Erkläre}, wie Hefe als Katalysator wirkt.
				\end{enumerate}
				
				
				\begin{tcolorbox}[enhanced,
					colback=white,
					colframe=darkgray,
					fonttitle=\sffamily\bfseries\large, 
					title=Informationstexte,  % search keyword::Informationstexte 
					attach boxed title to top left={xshift=3.2mm,yshift=-0.50mm},
					boxed title style={skin=enhancedfirst jigsaw,size=small,arc=1mm,bottom=-1mm,colframe=darkgray,height=0.75cm},
					colbacktitle=darkgray,
					drop lifted shadow]
					\begin{wrapfigure}{L}{0.15\textwidth}  
						\centering
						\vspace{-14pt}  % to align image with first line of text
						\includegraphics[width=0.9\textwidth]{symbols/symbol_tex_content}
					\end{wrapfigure}
					
					Ein Katalysator ist ein Stoff, der die Aktivierungsenergie senkt, die von den Teilchen für den Start einer chemische Reaktion benötigt wird. Im Diagram ist die Energie als rote Linie dargestellt\footnote{Quelle: https://de.wikipedia.org/wiki/Katalysator [Stand: 17.7.2020]}. Da der Katalysator die Energiemenge, die benötigt wird, senkt, erhöht er die Reaktionsgeschwindigkeit. Die Reaktion läuft viel schneller ab. Der Katalysator selbst wird während der Reaktion nicht verbraucht. Das bedeutet, dass man die gleiche Substanz mehrmals verwenden kann. \newline
					
					\begin{center}
						\includegraphics[width=0.5\textwidth]{images/energy_catalyst}
					\end{center}
					

					In der Natur arbeiten Enzyme oft als Katalysatoren. Sie haben die gleichen drei Eigenschaften wie chemische Katalysatoren. Während der Gärung wird Zucker in den reaktiven Zentren der Hefe in Alkohol und Kohlendioxid (Anaerober Prozess) umgewandelt. Ohne die als Katalysator wirkende Hefe würde diese chemische Reaktion nicht stattfinden.
				\end{tcolorbox}

% ***** start copy here *********************************************************
% *******************************************************************************
\vspace{1cm}	
			\begin{center}
				\begin{tcolorbox}[enhanced,
					width=0.75\textwidth,
					colback=white,
					colframe=blue,
					fonttitle=\sffamily\bfseries\large, 
					title=Verbindungen herstellen,  % search keyword::Informationstexte 
					attach boxed title to top left={xshift=3.2mm,yshift=-0.50mm},
					boxed title style={skin=enhancedfirst jigsaw,size=small,arc=1mm,bottom=-1mm,colframe=blue,height=0.75cm},
					colbacktitle=blue,
					drop lifted shadow]
					
					\begin{wrapfigure}{L}{0.15\textwidth}  
						\centering
						\vspace{-14pt}  % to align image with first line of text
						\includegraphics[width=0.7\textwidth]{symbols/symbol_tex_connect}
					\end{wrapfigure}
					
					\textit{Alle Kapitel sind miteinander verbunden. Wissen aus dem vorherigen Kapitel kann dir im nächsten Kapitel helften. Dazu musst du die Verbindungn zwischen den Kapiteln herstellen.} \newline
					
					\textbf{Auftrag: Nenne stichpunktartig drei wichtige Sachverhalte aus dem Kapitel, die du in Zukunft brauchen könntest.}
						
%					\begin{center}
%						%\vspace{1.1cm}
%						\noindent\rule{12cm}{0.2pt}
%						\vspace{1.1cm}
%						\noindent\rule{12cm}{0.1pt}
%						\vspace{1.1cm}
%						\noindent\rule{12cm}{0.1pt}
%						\vspace{1.1cm}
%						\noindent\rule{12cm}{0.1pt}
%						\vspace{1.1cm}
%						\noindent\rule{12cm}{0.1pt}
%						\vspace{0.1cm}
%					\end{center}

				\end{tcolorbox}
			\end{center}
			
% ***** end copy here *********************************************************
% *******************************************************************************	

\newpage

				% ***** start copy here *********************************************************
				% *******************************************************************************
				%            __ _           _   _                        _                _   
				%  _ __ ___ / _| | ___  ___| |_(_) ___  _ __         ___| |__   __ _ _ __| |_ 
				% | '__/ _ \ |_| |/ _ \/ __| __| |/ _ \| '_ \ _____ / __| '_ \ / _` | '__| __|
				% | | |  __/  _| |  __/ (__| |_| | (_) | | | |_____| (__| | | | (_| | |  | |_ 
				% |_|  \___|_| |_|\___|\___|\__|_|\___/|_| |_|      \___|_| |_|\__,_|_|   \__|
				%                                                                            
				%
				
				\vspace{0.3cm}
				\begin{center}
					\begin{tcolorbox}[enhanced,
						width=0.75\textwidth,
						colback=white,
						colframe=darkgray,
						fonttitle=\sffamily\bfseries\large, 
						title=Kurzreflexion,  % search keyword::Informationstexte 
						attach boxed title to top left={xshift=3.2mm,yshift=-0.50mm},
						boxed title style={skin=enhancedfirst jigsaw,size=small,arc=1mm,bottom=-1mm,colframe=darkgray,height=0.75cm},
						colbacktitle=darkgray,
						drop lifted shadow]
						
						% \begin{figure}[htbp]  % not good in ecolorbox
						\textbf{Auftrag: Bewerte deine Arbeit in der letzten Einheit selbst.} \newline
						\begin{center}
							\begin{tikzpicture}[scale=1]
								\path (0:0cm) coordinate (O); % define coordinate for origin
								
								% draw the spiderweb
								\foreach \X in {1,...,\D}{
									\draw (\X*\A:0) -- (\X*\A:\R);
								}
								
								\foreach \Y in {0,...,\U}{
									\foreach \X in {1,...,\D}{
								\path (\X*\A:\Y*\R/\U) coordinate (D\X-\Y);
								\fill (D\X-\Y) circle (1pt);
									}
									\draw [opacity=0.3] (0:\Y*\R/\U) \foreach \X in {1,...,\D}{
										-- (\X*\A:\Y*\R/\U)
									} -- cycle;
								}
								
								% define labels for each dimension axis (names config option)
								\path (1*\A:\L) node (L1) {\tiny Verständnis};
								\path (2*\A:\L) node (L2) {\tiny alle Aufgaben gelöst};
								\path (3*\A:\L) node (L3) {\tiny Zeit effektiv genutz};
								\path (4*\A:\L) node (L4) {\tiny Einzelarbeit};
								\path (5*\A:\L) node (L5) {\tiny Gruppenarbeit};
							
							\end{tikzpicture}
							% \caption{Diagramm Slebstreflexion}
							% \caption{Spiderweb Diagram (\D~Dimensions, \U-Notch Scale, 3 Samples)}
							% \label{fig:spiderweb}
							% \end{figure} 
						\end{center}
						\textbf{{\Large Was kannst du in der nächsten Stunde verbessern?}}
						\begin{center}
							%\vspace{1.1cm}
							\noindent\rule{12cm}{0.2pt}
							\vspace{1.1cm}
							\noindent\rule{12cm}{0.1pt}
							\vspace{1.1cm}
							\noindent\rule{12cm}{0.1pt}
							\vspace{1.1cm}
							\noindent\rule{12cm}{0.1pt}
							\vspace{1.1cm}
							\noindent\rule{12cm}{0.1pt}
							\vspace{1.1cm}
							\noindent\rule{12cm}{0.1pt}
							\vspace{1.1cm}
							\noindent\rule{12cm}{0.1pt}
							\vspace{1.1cm}
							\noindent\rule{12cm}{0.1pt}
						\end{center}
					\end{tcolorbox}
				\end{center}
								
				% *******************************************************************************
				% ***** end copy here ***********************************************************

				
\newpage

		\section{Destillation}

			\textit{Super! Du hast wohl Wein hergestellt! Aber gegen das Virus (und gegen Bakterien) brauchst du höherkonzentrierten Alkohol. Mit Wein ist da nichts zu machen. Aber wie kann man den Alkoholgehalt auf über 60\% erhöhen?} \newline
			
			\begin{minipage}{0.7\textwidth}
				\noindent \textbf{Am Ende dieses Kapitels sollst du ... :}
				\begin{enumerate}
					\item ... von den Eigenschaften des Ethanols auf dessen Verwendungsmöglichkeiten zu schließen.
					\item ... die Destillation als Trennverfahren durchführen und erläutern können.
				\end{enumerate}
				\textbf{Vorgehensweise:}
				\begin{enumerate}
					\item Arbeite in Gruppen von 4-5 SuS.
					\item Führt das Experiment durch.
					\item Bearbeite anschließend die restlichen Arbeitsaufträge.
					\item Nachdem du den Arbeitsabschnitt erledigt hast, tausche dich mit einigen MitschülerInnen aus. 
				\end{enumerate}
				
			\end{minipage}
			\hspace{0.1\textwidth}
			\begin{minipage}{0.2\textwidth}
				\begin{tcolorbox}
					[enhanced,
					width=0.9\textwidth,
					colback=white,
					colframe=black,
					fonttitle=\sffamily\bfseries\large, 
					title=Zeit,  % search keyword::Zeit
					attach boxed title to top center={xshift=-0.0mm,yshift=-0.50mm},
					boxed title style={skin=enhancedfirst jigsaw,size=small,arc=1mm,bottom=-1mm,colframe=black,height=0.75cm},
					colbacktitle=black,
					drop lifted shadow]
					\centering
					\includegraphics[width=0.9\textwidth]{symbols/symbol_tex_time}
					
					\begin{center}
						\textbf{45min}
					\end{center}
				\end{tcolorbox}
			\end{minipage}
			
			\begin{center}
				\noindent\rule{18cm}{0.1pt}
			\end{center}

\newpage

		\subsection{Die Schnelldestillation von Ethanol}
		
			\noindent \textbf{Auftrag: Trenne das entstandene Ethanol von der restlichen Flüssigkeit deines Gärversuchs.}
			
			\begin{enumerate}
				\item \textbf{Lies} die Durchführung und führe das \textbf{Experiment} durch.
				\item \textbf{Besprich die Durchführung und das Experiment mit der Lehrkraft! SCHUTZBRILLE!}
				\item \textbf{Protokolliere} das Experiment. Erstelle auch eine Materialliste.
				\item Beantworte die gegebenen \textbf{Fragen} als Teil deiner \textbf{Auswertung}.
			\end{enumerate}
		
			\begin{tcolorbox}[enhanced,
				colback=white,
				colframe=green!30!black,
				fonttitle=\sffamily\bfseries\large, 
				title=Durchführung,  % search keyword::Durchführung
				attach boxed title to top left={xshift=3.2mm,yshift=-0.50mm},
				boxed title style={skin=enhancedfirst jigsaw,size=small,arc=1mm,bottom=-1mm,colframe=green!50!black,height=0.75cm},
				colbacktitle=green!50!black,
				drop lifted shadow]
				\begin{wrapfigure}{L}{0.15\textwidth}  
					\centering
					\vspace{-14pt}  % to align image with first line of text
					\includegraphics[width=0.7\textwidth]{symbols/symbol_tex_method}
				\end{wrapfigure}
				
					Fülle deinen Wein einen Finger hoch in ein 100ml Erlenmeyerkolben und füge einige Siedesteine hinzu. Verschließe den Kolben mit dem Stopfen mit einem 30cm Glasrohr. Stell das Ganze auf das Stativ. Erhitze den Kolben mit dem Bunsenbrenner langsam, bis es kocht. Verwende die helle Flamme. Du kannst es auch auf einer Kochplatte langsam direkt erhitzen, wenn du keinen Bunsenbrenner hast. Wenn der Wein kocht, nimm den Bunsenbrenner weg. Lass etwas Gas entweichen. Halte dann ein angezündetes Streichholz an das Ende des Glasrohrs.
					
					\begin{center}
						\includegraphics{images/destillation_method}
					\end{center}
			\end{tcolorbox}

			\vspace{0.3cm}
			\noindent \textbf{Fragen für die Auswertung.}
			\begin{enumerate}
				\item \textbf{Erkläre} deine Beobachtungen.
				\item \textbf{Erkläre}, warum nur Ethanol durch das Glasrohr aufsteigt.
				\item Der Ethanol soll als Desinfektionsmittel dienen und muss abgefüllt werden. \textbf{Beschreibe}, wie du den Versuchsaufbau verändern musst, um flüssiges Ethanol zu erhalten.
				\item \textbf{Recherchiere} und \textbf{erkläre}, warum hochkonzentriertes Ethanol gegen Viren und Bakterien wirksam ist.
			\end{enumerate}			


% ***** start copy here *********************************************************
% *******************************************************************************
\vspace{1cm}	
			\begin{center}
				\begin{tcolorbox}[enhanced,
					width=0.75\textwidth,
					colback=white,
					colframe=blue,
					fonttitle=\sffamily\bfseries\large, 
					title=Verbindungen herstellen,  % search keyword::Informationstexte 
					attach boxed title to top left={xshift=3.2mm,yshift=-0.50mm},
					boxed title style={skin=enhancedfirst jigsaw,size=small,arc=1mm,bottom=-1mm,colframe=blue,height=0.75cm},
					colbacktitle=blue,
					drop lifted shadow]
					
					\begin{wrapfigure}{L}{0.15\textwidth}  
						\centering
						\vspace{-14pt}  % to align image with first line of text
						\includegraphics[width=0.7\textwidth]{symbols/symbol_tex_connect}
					\end{wrapfigure}
					
					\textit{Alle Kapitel sind miteinander verbunden. Wissen aus dem vorherigen Kapitel kann dir im nächsten Kapitel helften. Dazu musst du die Verbindungn zwischen den Kapiteln herstellen.} \newline
					
					\textbf{Auftrag: Nenne stichpunktartig drei wichtige Sachverhalte aus dem Kapitel, die du in Zukunft brauchen könntest.}
						
%					\begin{center}
%						%\vspace{1.1cm}
%						\noindent\rule{12cm}{0.2pt}
%						\vspace{1.1cm}
%						\noindent\rule{12cm}{0.1pt}
%						\vspace{1.1cm}
%						\noindent\rule{12cm}{0.1pt}
%						\vspace{1.1cm}
%						\noindent\rule{12cm}{0.1pt}
%						\vspace{1.1cm}
%						\noindent\rule{12cm}{0.1pt}
%						\vspace{0.1cm}
%					\end{center}

				\end{tcolorbox}
			\end{center}
			
% ***** end copy here *********************************************************
% *******************************************************************************

\newpage

				% ***** start copy here *********************************************************
				% *******************************************************************************
				%            __ _           _   _                        _                _   
				%  _ __ ___ / _| | ___  ___| |_(_) ___  _ __         ___| |__   __ _ _ __| |_ 
				% | '__/ _ \ |_| |/ _ \/ __| __| |/ _ \| '_ \ _____ / __| '_ \ / _` | '__| __|
				% | | |  __/  _| |  __/ (__| |_| | (_) | | | |_____| (__| | | | (_| | |  | |_ 
				% |_|  \___|_| |_|\___|\___|\__|_|\___/|_| |_|      \___|_| |_|\__,_|_|   \__|
				%                                                                            
				%
				
				\vspace{0.3cm}
				\begin{center}
					\begin{tcolorbox}[enhanced,
						width=0.75\textwidth,
						colback=white,
						colframe=darkgray,
						fonttitle=\sffamily\bfseries\large, 
						title=Kurzreflexion,  % search keyword::Informationstexte 
						attach boxed title to top left={xshift=3.2mm,yshift=-0.50mm},
						boxed title style={skin=enhancedfirst jigsaw,size=small,arc=1mm,bottom=-1mm,colframe=darkgray,height=0.75cm},
						colbacktitle=darkgray,
						drop lifted shadow]
						
						% \begin{figure}[htbp]  % not good in ecolorbox
						\textbf{Auftrag: Bewerte deine Arbeit in der letzten Einheit selbst.} \newline
						\begin{center}
							\begin{tikzpicture}[scale=1]
								\path (0:0cm) coordinate (O); % define coordinate for origin
								
								% draw the spiderweb
								\foreach \X in {1,...,\D}{
									\draw (\X*\A:0) -- (\X*\A:\R);
								}
								
								\foreach \Y in {0,...,\U}{
									\foreach \X in {1,...,\D}{
								\path (\X*\A:\Y*\R/\U) coordinate (D\X-\Y);
								\fill (D\X-\Y) circle (1pt);
									}
									\draw [opacity=0.3] (0:\Y*\R/\U) \foreach \X in {1,...,\D}{
										-- (\X*\A:\Y*\R/\U)
									} -- cycle;
								}
								
								% define labels for each dimension axis (names config option)
								\path (1*\A:\L) node (L1) {\tiny Verständnis};
								\path (2*\A:\L) node (L2) {\tiny alle Aufgaben gelöst};
								\path (3*\A:\L) node (L3) {\tiny Zeit effektiv genutz};
								\path (4*\A:\L) node (L4) {\tiny Einzelarbeit};
								\path (5*\A:\L) node (L5) {\tiny Gruppenarbeit};
							
							\end{tikzpicture}
							% \caption{Diagramm Slebstreflexion}
							% \caption{Spiderweb Diagram (\D~Dimensions, \U-Notch Scale, 3 Samples)}
							% \label{fig:spiderweb}
							% \end{figure} 
						\end{center}
						\textbf{{\Large Was kannst du in der nächsten Stunde verbessern?}}
						\begin{center}
							%\vspace{1.1cm}
							\noindent\rule{12cm}{0.2pt}
							\vspace{1.1cm}
							\noindent\rule{12cm}{0.1pt}
							\vspace{1.1cm}
							\noindent\rule{12cm}{0.1pt}
							\vspace{1.1cm}
							\noindent\rule{12cm}{0.1pt}
							\vspace{1.1cm}
							\noindent\rule{12cm}{0.1pt}
							\vspace{1.1cm}
							\noindent\rule{12cm}{0.1pt}
							\vspace{1.1cm}
							\noindent\rule{12cm}{0.1pt}
							\vspace{1.1cm}
							\noindent\rule{12cm}{0.1pt}
						\end{center}
					\end{tcolorbox}
				\end{center}
								
				% *******************************************************************************
				% ***** end copy here ***********************************************************


\newpage

		\section{Die Eigenschaften von Ethanol}

			\textit{Alkohol herzustellen ist einfach - oder zumindest sieht es einfach aus. Aber woher wissen wir, dass wir tatsächlich Alkohol produziert haben? Wie können wir Alkohole identifizieren? Und gibt es einen Unterschied zwischen Alkohol zum Trinken und den Chemikalien, die als Alkohole bezeichnet werden?} \newline
			
			\begin{minipage}{0.7\textwidth}
				\noindent \textbf{Am Ende dieses Kapitels solltest du ... :}
				\begin{enumerate}
					\item ... beschreiben können, dass Alkohole (auch Alkanole genannt) Verbindungen sind, die aus Kohlenstoffatomen, Wasserstoffatomen und Sauerstoffatomen aufgebaut sind.
					\item ... die Hydroxylgruppe (OH-Gruppe) als funktionelle Gruppe aller Alkohole benennen können.
					\item ... die Strukturformeln der ersten zehn Alkohole aufzuzeichnen (Methanol bis Decanol) können. 
					\item ... erklären können, was die homologe Reihe der Alkohole (Alkanole) ist.
					\item ... die Definitionen für die Begriffe Van-der-Waals-Kräfte und Wasserstoffbrücken-bindungen nennen können.
					\item ... die Definitionen für die Begriffe \textit{polar} und \textit{unpolar} nennen können.
					\item ... beschreiben können, wie sich die Siedepunkte der Alkane und Alkohole unterscheiden.
					\item ... erklären können, welche zwischenmolekularen Kräfte zwischen Alkoholmolekülen wirken.
					\item ... die Zuordnung eines Moleküls zur Gruppe der Alkanole mit dem Vorhandensein der funktionellen Gruppe begründen können.
				\end{enumerate}	
			
				\noindent \textbf{Vorgehensweise:}
				\begin{enumerate}
					\item Arbeite in Gruppen von 4-5 SuS.
					\item Bearbeite die Arbeitsaufträge für die Experimente gemeinsam!
					\item Schreibe ein \textbf{eigenes} Protokoll!
					\item Frage die Lehrkraft nach einer Vorlage für ein digitales Protokoll.
				\end{enumerate}	
			\end{minipage}
			\hspace{0.1\textwidth}
			\begin{minipage}{0.2\textwidth}
				\begin{tcolorbox}
					[enhanced,
					width=0.9\textwidth,
					colback=white,
					colframe=black,
					fonttitle=\sffamily\bfseries\large, 
					title=Zeit,  % search keyword::Zeit
					attach boxed title to top center={xshift=-0.0mm,yshift=-0.50mm},
					boxed title style={skin=enhancedfirst jigsaw,size=small,arc=1mm,bottom=-1mm,colframe=black,height=0.75cm},
					colbacktitle=black,
					drop lifted shadow]
					\centering
					\includegraphics[width=0.9\textwidth]{symbols/symbol_tex_time}
					
					\begin{center}
						\textbf{90min}
					\end{center}
				\end{tcolorbox}
			\end{minipage}
			
			\vspace{0.3cm}
			\begin{tcolorbox}[enhanced,
				colback=white,
				colframe=orange!60!red,
				fonttitle=\sffamily\bfseries\large, 
				title=Wiederholung,  % search keyword::Wiederholung
				attach boxed title to top left={xshift=3.2mm,yshift=-0.50mm},
				boxed title style={skin=enhancedfirst jigsaw,size=small,arc=1mm,bottom=-1mm,colframe=orange!60!red,height=0.75cm},
				colbacktitle=orange!60!red,
				% sharp corners,
				drop lifted shadow]	
				\begin{wrapfigure}{L}{0.15\textwidth}  
					\centering
					\vspace{-14pt}  % to align image with first line of text
					\includegraphics[width=0.9\textwidth]{symbols/symbol_tex_review}
				\end{wrapfigure}
				
				\textbf{Nenne} ein polares und ein unpolares Lösungsmittel. \textbf{Erläutere}, mit Hilfe des \textit{Struktur-Kräfte-Eigenschaften-Konzepts}, warum es sich jeweils um ein polares bzw. unpolares Lösungsmittel handelt. \textbf{Nenne} dabei auch die Kräfte, die zwischen den Molekülen wirken.
				\vspace{0.3cm}
			\end{tcolorbox}
			
			\begin{center}
				\noindent\rule{18cm}{0.1pt}
			\end{center}
				
\newpage

			\subsection{Die experimentelle Untersuchung von Ethanol}
			
				\noindent \textbf{Auftrag: Untersuche experimentell die Eigenschaften von Ethanol.}
				\begin{enumerate}
					\item Untersuche verschiedene chemisch-physikalische Eigenschaften. \textit{Mindestens} musst du die folgenden Eigenschaften untersuchen:
						\begin{itemize}
							\item Brennbarkeit
							\item Löslichkeit in polaren und unpolaren Lösungsmitteln
							\item Siedetemperatur (in Absprache mit der Lehrkraft)
						\end{itemize}
					\item Hinweise zum Protokoll:
						\begin{itemize}
							\item Gliederung: Ein Protokoll mit mehreren Unterpunkten (Exp.1, Exp.2, ...) pro Gliederungspunkt (Material, Durchführung, ...)
							\item Du brauchst keine Hypothese zu schreiben! Aber wenn du die Vorüberlegungen durchgearbeitet hast, solltest du schon einige Hypothesen haben.
							\item  Als Teil deiner Auswertung beantworte auch die gegebenen Fragen(s.u.)! Schreibe dazu einen Fließtext, der alles erläutert.
							\item In deiner Auswertung muss du dich immer auf das Struktur-Kräfte-Eigenschaftskonzept beziehen!
							\item Abgabe: Digital (pdf) oder Ausdruck!
							\item \textbf{Deadline nach Absprache mit Lehrkraft.}
						\end{itemize}
					\item Weiterführende Fragen für die Auswertung:
						\begin{itemize}
							\item \textbf{Recherchiere} und \textbf{vergleiche} die Siede- und Schmelztemperaturen der Alkohole ($ C_{1} $-$ C_{10} $). \textbf{Zeichne} dazu ein Diagramm mit der Hilfe von Excel (oder vergleichbarerer Software) in dein Protokoll.
							\item \textbf{Erläutere}, an welchen chemischen Strukturen man Alkohole identifizieren kann. Zeichne die Struktur dabei ausführlich.
							\item \textbf{Formuliere die Reaktionsgleichung} für die Verbrennung von Ethanol. % neu für Schuljahr 2022/23
							\item \textbf{Recherchiere} die Siede- und Schmelztemperatur von Ethanol und Ethan und \textbf{erläutere} den Unterschied! Beziehe dich dabei auf das \textbf{Struktur-Kräfte-Eigenschaften-Konzept}.
							\item \textbf{Recherchiere} die Viskosität von Wasser und Ethanol und \textbf{erläutere} die Unterschiede!
							\item Ethanol wird als Frostschutzmittel genutzt. \textbf{Erläutere} warum!
						\end{itemize}
				\end{enumerate}

% ***** start copy here *********************************************************
% *******************************************************************************
\vspace{3cm}	
			\begin{center}
				\begin{tcolorbox}[enhanced,
					width=0.75\textwidth,
					colback=white,
					colframe=blue,
					fonttitle=\sffamily\bfseries\large, 
					title=Verbindungen herstellen,  % search keyword::Informationstexte 
					attach boxed title to top left={xshift=3.2mm,yshift=-0.50mm},
					boxed title style={skin=enhancedfirst jigsaw,size=small,arc=1mm,bottom=-1mm,colframe=blue,height=0.75cm},
					colbacktitle=blue,
					drop lifted shadow]
					
					\begin{wrapfigure}{L}{0.15\textwidth}  
						\centering
						\vspace{-14pt}  % to align image with first line of text
						\includegraphics[width=0.7\textwidth]{symbols/symbol_tex_connect}
					\end{wrapfigure}
					
					\textit{Alle Kapitel sind miteinander verbunden. Wissen aus dem vorherigen Kapitel kann dir im nächsten Kapitel helften. Dazu musst du die Verbindungn zwischen den Kapiteln herstellen.} \newline
					
					\textbf{Auftrag: Nenne stichpunktartig drei wichtige Sachverhalte aus dem Kapitel, die du in Zukunft brauchen könntest.}
						
%					\begin{center}
%						%\vspace{1.1cm}
%						\noindent\rule{12cm}{0.2pt}
%						\vspace{1.1cm}
%						\noindent\rule{12cm}{0.1pt}
%						\vspace{1.1cm}
%						\noindent\rule{12cm}{0.1pt}
%						\vspace{1.1cm}
%						\noindent\rule{12cm}{0.1pt}
%						\vspace{1.1cm}
%						\noindent\rule{12cm}{0.1pt}
%						\vspace{0.1cm}
%					\end{center}

				\end{tcolorbox}
			\end{center}
			
% ***** end copy here *********************************************************
% *******************************************************************************

\newpage

				% ***** start copy here *********************************************************
				% *******************************************************************************
				%            __ _           _   _                        _                _   
				%  _ __ ___ / _| | ___  ___| |_(_) ___  _ __         ___| |__   __ _ _ __| |_ 
				% | '__/ _ \ |_| |/ _ \/ __| __| |/ _ \| '_ \ _____ / __| '_ \ / _` | '__| __|
				% | | |  __/  _| |  __/ (__| |_| | (_) | | | |_____| (__| | | | (_| | |  | |_ 
				% |_|  \___|_| |_|\___|\___|\__|_|\___/|_| |_|      \___|_| |_|\__,_|_|   \__|
				%                                                                            
				%
				
				\vspace{0.3cm}
				\begin{center}
					\begin{tcolorbox}[enhanced,
						width=0.75\textwidth,
						colback=white,
						colframe=darkgray,
						fonttitle=\sffamily\bfseries\large, 
						title=Kurzreflexion,  % search keyword::Informationstexte 
						attach boxed title to top left={xshift=3.2mm,yshift=-0.50mm},
						boxed title style={skin=enhancedfirst jigsaw,size=small,arc=1mm,bottom=-1mm,colframe=darkgray,height=0.75cm},
						colbacktitle=darkgray,
						drop lifted shadow]
						
						% \begin{figure}[htbp]  % not good in ecolorbox
						\textbf{Auftrag: Bewerte deine Arbeit in der letzten Einheit selbst.} \newline
						\begin{center}
							\begin{tikzpicture}[scale=1]
								\path (0:0cm) coordinate (O); % define coordinate for origin
								
								% draw the spiderweb
								\foreach \X in {1,...,\D}{
									\draw (\X*\A:0) -- (\X*\A:\R);
								}
								
								\foreach \Y in {0,...,\U}{
									\foreach \X in {1,...,\D}{
								\path (\X*\A:\Y*\R/\U) coordinate (D\X-\Y);
								\fill (D\X-\Y) circle (1pt);
									}
									\draw [opacity=0.3] (0:\Y*\R/\U) \foreach \X in {1,...,\D}{
										-- (\X*\A:\Y*\R/\U)
									} -- cycle;
								}
								
								% define labels for each dimension axis (names config option)
								\path (1*\A:\L) node (L1) {\tiny Verständnis};
								\path (2*\A:\L) node (L2) {\tiny alle Aufgaben gelöst};
								\path (3*\A:\L) node (L3) {\tiny Zeit effektiv genutz};
								\path (4*\A:\L) node (L4) {\tiny Einzelarbeit};
								\path (5*\A:\L) node (L5) {\tiny Gruppenarbeit};
							
							\end{tikzpicture}
							% \caption{Diagramm Slebstreflexion}
							% \caption{Spiderweb Diagram (\D~Dimensions, \U-Notch Scale, 3 Samples)}
							% \label{fig:spiderweb}
							% \end{figure} 
						\end{center}
						\textbf{{\Large Was kannst du in der nächsten Stunde verbessern?}}
						\begin{center}
							%\vspace{1.1cm}
							\noindent\rule{12cm}{0.2pt}
							\vspace{1.1cm}
							\noindent\rule{12cm}{0.1pt}
							\vspace{1.1cm}
							\noindent\rule{12cm}{0.1pt}
							\vspace{1.1cm}
							\noindent\rule{12cm}{0.1pt}
							\vspace{1.1cm}
							\noindent\rule{12cm}{0.1pt}
							\vspace{1.1cm}
							\noindent\rule{12cm}{0.1pt}
							\vspace{1.1cm}
							\noindent\rule{12cm}{0.1pt}
							\vspace{1.1cm}
							\noindent\rule{12cm}{0.1pt}
						\end{center}
					\end{tcolorbox}
				\end{center}
								
				% *******************************************************************************
				% ***** end copy here ***********************************************************



\newpage

		\section{Die Nomenklatur der Alkohole}

			\textit{Während du dich vor Zombies und virenverseuchten Menschen schützt, fällt dir auf, dass nicht alle Desinfektionsmittel mit Ethanol sind. Aber was ist Isopropanol? Oder 2-Propanol? Wie bei den Alkanen gibt es eine Vielzahl von Möglichkeiten, die Atome und funktionellen Gruppen zu Molekülen zu kombinieren. Daher müssen wir uns die Nomenklatur der Alkohole genauer ansehen!} \newline
			
			\begin{minipage}{0.7\textwidth}
				\noindent \textbf{Am Ende dieses Kapitels solltest du ... :}
				\begin{enumerate}
				    \item ... erklären können, dass Polyalkohole Alkanole mit mehr als einer funktionellen Gruppe sind.
				    \item ... den Unterschied zwischen primären, sekundären und tertiären Alkoholen mit Hilfe von Strukturformeln zu erklären. 
				\end{enumerate}	
				
				\noindent \textbf{Vorgehensweise:}
				\begin{enumerate}
					\item Bearbeite die Arbeitsaufträge. 
					\item Vergleiche deine Ergebnisse mit deinen MitschülerInnen.
				\end{enumerate}	
			\end{minipage}
			\hspace{0.1\textwidth}
			\begin{minipage}{0.2\textwidth}
				\begin{tcolorbox}
					[enhanced,
					width=0.9\textwidth,
					colback=white,
					colframe=black,
					fonttitle=\sffamily\bfseries\large, 
					title=Zeit,  % search keyword::Zeit
					attach boxed title to top center={xshift=-0.0mm,yshift=-0.50mm},
					boxed title style={skin=enhancedfirst jigsaw,size=small,arc=1mm,bottom=-1mm,colframe=black,height=0.75cm},
					colbacktitle=black,
					drop lifted shadow]
					\centering
					\includegraphics[width=0.9\textwidth]{symbols/symbol_tex_time}
					
					\begin{center}
						\textbf{90min}
					\end{center}
				\end{tcolorbox}
			\end{minipage}
			
			\vspace{0.3cm}
			\begin{tcolorbox}[enhanced,
				colback=white,
				colframe=orange!60!red,
				fonttitle=\sffamily\bfseries\large, 
				title=Wiederholung,  % search keyword::Wiederholung
				attach boxed title to top left={xshift=3.2mm,yshift=-0.50mm},
				boxed title style={skin=enhancedfirst jigsaw,size=small,arc=1mm,bottom=-1mm,colframe=orange!60!red,height=0.75cm},
				colbacktitle=orange!60!red,
				% sharp corners,
				drop lifted shadow]	
				\begin{wrapfigure}{L}{0.15\textwidth}  
					\centering
					\vspace{-14pt}  % to align image with first line of text
					\includegraphics[width=0.9\textwidth]{symbols/symbol_tex_review}
				\end{wrapfigure}
				
				\textbf{Erläutere} die wichtigsten Regeln zur Benennung der Alkane. 
				\vspace{1.6cm}
			\end{tcolorbox}
			
			\begin{center}
				\noindent\rule{18cm}{0.1pt}
			\end{center}
						
\newpage						
			\subsection{Nomenklatur und Isomere}
			
				\noindent \textbf{Auftrag: Erkläre, wie man die verschiedenen Alkohole benennt.}
				\begin{enumerate}
					\item \textbf{Lies} den Text.
					\item Vervollständige die erste Tabelle. \textbf{Erkläre} den Unterschied zwischen primären, sekundäre und tertiären Alkoholen.
					\item Vervollständige die zweite Tabelle. \textbf{Erkläre} den Unterschied zwischen einwertigen, zweiwertigen und dreiwertigen Alkoholen.
				\end{enumerate}
				
				
				
				\begin{tcolorbox}[enhanced,
					colback=white,
					colframe=darkgray,
					fonttitle=\sffamily\bfseries\large, 
					title=Informationstexte,  % search keyword::Informationstexte 
					attach boxed title to top left={xshift=3.2mm,yshift=-0.50mm},
					boxed title style={skin=enhancedfirst jigsaw,size=small,arc=1mm,bottom=-1mm,colframe=darkgray,height=0.75cm},
					colbacktitle=darkgray,
					drop lifted shadow]
					\begin{wrapfigure}{L}{0.15\textwidth}  
						\centering
						\vspace{-14pt}  % to align image with first line of text
						\includegraphics[width=0.9\textwidth]{symbols/symbol_tex_content}
					\end{wrapfigure}
					
					In der Chemie werden alle Verbindungen, die eine oder mehrere Hydroxyl-Gruppen (-OH) haben, als Alkohole bezeichnet. Die Hydroxylgruppe wird auch als funktionelle Gruppe der Alkohole bezeichnet. Die gebräuchliche Formel für Alkohole lautet $ C_{n}H_{2n+1}OH $.  \newline
					\textbf{Isomere} \newline
					Beim "Bau" der LEWIS-Formel von \ch{C3H7OH} würden Sie zwei verschiedene Versionen vorfinden. In einem Fall steht die OH-Gruppe am Anfang (oder Ende) der C-Kette. Diese Substanz wird n-Propanol genannt. In einem anderen Fall befindet sich die Hydroxyl-Gruppe in der Mitte der C-Kette. Diese Substanz wird als Iso-Propanol bezeichnet. Beide Substanzen werden Isomere genannt. Sie haben die gleiche (numerische) chemische Formel, aber eine unterschiedliche Molekularstruktur. Butanol hat vier Isomere, da auch die C-Kette verändert werden kann.  \newline
					\textbf{Nomenklatur} \newline
					Der wissenschaftliche Name für n-Propanol ist Propan-1-ol, Iso-Propanol wird als Propan-2-ol bezeichnet. Wie bei der Nomenklatur der Alkane beginnt man mit der längsten C-Kette. Danach muss man die Position der OH-Gruppe bestimmen. \newline
					\textbf{Verschiedene Arten von Alkoholen} \newline
					Je nachdem, wie viele andere C-Atome mit der OH-Gruppe an das C-Atom gebunden sind, werden die verschiedenen Isomere als primäre (z.B. n-Propan-1-ol), sekundäre (z.B. Propan-2-ol) oder tertiäre Alkohole (z.B. 2-Methylpropan-2ol) bezeichnet. \newline
					Wenn es mehrere OH-Gruppen gibt, führen sie diese ebenfalls im Namen auf. Alkohole mit einer OH-Gruppe werden als einwertig, mit zwei OH-Gruppen als zweiwertig und mit drei OH-Gruppen als dreiwertig bezeichnet.					
				\end{tcolorbox}

% TODO: Hier macht Propanol keinen Sinn, besser Pentanol wegen Psotion der OH-Gruppe
				% \vspace{0.3cm}
				\begin{center}
					\begin{tabular}{|c|c|c|c|}
						\hline
						\textbf{Name} & Propan-1-ol & ... & Propan-3-ol \\
						\hline
						\textbf{Struktur} & ... & \chemfig{H-C(-[2]H)(-[6]H)-C(-[2]H)(-[6]OH)-C(-[2]H)(-[6]H)-H} & ... \\
						\hline 
					\end{tabular}
				\end{center}

				\begin{center}
					\begin{tabular}{|c|c|c|c|}
						\hline
						\textbf{Name} & Propan-1-ol (einwertig) & Propan-1,2-diol (2-wertig) & Propan-1,2,3-triol (3-wertig) \\
						\hline
						\textbf{Struktur} & ... & ... & ... \\
						\hline 
					\end{tabular}
				\end{center}


				\vspace{0.3cm}
				\textbf{Auftrag: Übung macht den Meister/die Meisterin!}
				\begin{enumerate}
					\item \textbf{Zeichne} als möglichen LEWIS-Formeln von \ch{C4H9OH}.
						\begin{itemize}
							\item \textbf{Benenne} alle Moleküle.
							\item \textbf{Erläutere}, welche der Moleküle ein-, zwei- oder dreiwertig sind.
							\item \textbf{Erläutere} an einem Beispiel, wie man einen einwertigen Alkohol in einen zweiwertigen Alkohol umwandeln kann.
						\end{itemize}
					\item \textbf{Benenne} die folgenden Moleküle und \textbf{erkläre}, wie du zu dem Namen gekommen bist.
						\begin{itemize}
							\item \chemfig{H-[,1.5]C(-[2]H)(-[6]H)-[,1.5]C(-[2]H)(-[6]C(-[4,0.75]H)(-[6,0.75]H)(-[8,0.75]H))-[,1.5]C(-[2]H)(-[6]OH)-[,1.5]C(-[2]H)(-[6]H)-[,1.5]C(-[2]H)(-[6]H)-H}
							\item \chemfig{CH_3-C(-[2]CH_3)(-[6]CH_3)-CH_2-CH(-[6]C_2H_5)-CH_2-CH_2-OH}
						\end{itemize}
				\end{enumerate}
				
				\vspace{0.3cm}
				\begin{tcolorbox}[enhanced,
					colback=white,
					colframe=red,
					fonttitle=\sffamily\bfseries\large, 
					title=Für schnelle Schüler\_innen,  % search keyword::schnelle_Schüler
					attach boxed title to top left={xshift=3.2mm,yshift=-0.40mm},
					boxed title style={skin=enhancedfirst jigsaw,size=small,arc=1mm,bottom=-1mm,colframe=red,height=0.75cm},
					colbacktitle=red,
					drop lifted shadow]
					\begin{wrapfigure}{L}{0.15\textwidth}  
						\centering
						\vspace{-14pt}  % to align image with first line of text
						\includegraphics[width=0.8\textwidth]{symbols/symbol_tex_faststudents}
					\end{wrapfigure}
					
					Bildet Gruppen von max. 4 Lernenden. Jede(r) Lernende(r) zeichnet ein Alkoholmolekül mit max. 10 C-Atomen und benennt es. Zeigt danach die LEWIS-Formel den anderen Gruppenmitgliedern, damit sie das Molekül benennen können. Vergleicht eure Ergebnisse!
					\vspace{0.3cm}  % to fill empty space in tcolorbox
				\end{tcolorbox}

% ***** start copy here *********************************************************
% *******************************************************************************
\vspace{9cm}	
			\begin{center}
				\begin{tcolorbox}[enhanced,
					width=0.75\textwidth,
					colback=white,
					colframe=blue,
					fonttitle=\sffamily\bfseries\large, 
					title=Verbindungen herstellen,  % search keyword::Informationstexte 
					attach boxed title to top left={xshift=3.2mm,yshift=-0.50mm},
					boxed title style={skin=enhancedfirst jigsaw,size=small,arc=1mm,bottom=-1mm,colframe=blue,height=0.75cm},
					colbacktitle=blue,
					drop lifted shadow]
					
					\begin{wrapfigure}{L}{0.15\textwidth}  
						\centering
						\vspace{-14pt}  % to align image with first line of text
						\includegraphics[width=0.7\textwidth]{symbols/symbol_tex_connect}
					\end{wrapfigure}
					
					\textit{Alle Kapitel sind miteinander verbunden. Wissen aus dem vorherigen Kapitel kann dir im nächsten Kapitel helften. Dazu musst du die Verbindungn zwischen den Kapiteln herstellen.} \newline
					
					\textbf{Auftrag: Nenne stichpunktartig drei wichtige Sachverhalte aus dem Kapitel, die du in Zukunft brauchen könntest.}
						
%					\begin{center}
%						%\vspace{1.1cm}
%						\noindent\rule{12cm}{0.2pt}
%						\vspace{1.1cm}
%						\noindent\rule{12cm}{0.1pt}
%						\vspace{1.1cm}
%						\noindent\rule{12cm}{0.1pt}
%						\vspace{1.1cm}
%						\noindent\rule{12cm}{0.1pt}
%						\vspace{1.1cm}
%						\noindent\rule{12cm}{0.1pt}
%						\vspace{0.1cm}
%					\end{center}

				\end{tcolorbox}
			\end{center}
			
% ***** end copy here *********************************************************
% *******************************************************************************

\newpage

				% ***** start copy here *********************************************************
				% *******************************************************************************
				%            __ _           _   _                        _                _   
				%  _ __ ___ / _| | ___  ___| |_(_) ___  _ __         ___| |__   __ _ _ __| |_ 
				% | '__/ _ \ |_| |/ _ \/ __| __| |/ _ \| '_ \ _____ / __| '_ \ / _` | '__| __|
				% | | |  __/  _| |  __/ (__| |_| | (_) | | | |_____| (__| | | | (_| | |  | |_ 
				% |_|  \___|_| |_|\___|\___|\__|_|\___/|_| |_|      \___|_| |_|\__,_|_|   \__|
				%                                                                            
				%
				
				\vspace{0.3cm}
				\begin{center}
					\begin{tcolorbox}[enhanced,
						width=0.75\textwidth,
						colback=white,
						colframe=darkgray,
						fonttitle=\sffamily\bfseries\large, 
						title=Kurzreflexion,  % search keyword::Informationstexte 
						attach boxed title to top left={xshift=3.2mm,yshift=-0.50mm},
						boxed title style={skin=enhancedfirst jigsaw,size=small,arc=1mm,bottom=-1mm,colframe=darkgray,height=0.75cm},
						colbacktitle=darkgray,
						drop lifted shadow]
						
						% \begin{figure}[htbp]  % not good in ecolorbox
						\textbf{Auftrag: Bewerte deine Arbeit in der letzten Einheit selbst.} \newline
						\begin{center}
							\begin{tikzpicture}[scale=1]
								\path (0:0cm) coordinate (O); % define coordinate for origin
								
								% draw the spiderweb
								\foreach \X in {1,...,\D}{
									\draw (\X*\A:0) -- (\X*\A:\R);
								}
								
								\foreach \Y in {0,...,\U}{
									\foreach \X in {1,...,\D}{
								\path (\X*\A:\Y*\R/\U) coordinate (D\X-\Y);
								\fill (D\X-\Y) circle (1pt);
									}
									\draw [opacity=0.3] (0:\Y*\R/\U) \foreach \X in {1,...,\D}{
										-- (\X*\A:\Y*\R/\U)
									} -- cycle;
								}
								
								% define labels for each dimension axis (names config option)
								\path (1*\A:\L) node (L1) {\tiny Verständnis};
								\path (2*\A:\L) node (L2) {\tiny alle Aufgaben gelöst};
								\path (3*\A:\L) node (L3) {\tiny Zeit effektiv genutz};
								\path (4*\A:\L) node (L4) {\tiny Einzelarbeit};
								\path (5*\A:\L) node (L5) {\tiny Gruppenarbeit};
							
							\end{tikzpicture}
							% \caption{Diagramm Slebstreflexion}
							% \caption{Spiderweb Diagram (\D~Dimensions, \U-Notch Scale, 3 Samples)}
							% \label{fig:spiderweb}
							% \end{figure} 
						\end{center}
						\textbf{{\Large Was kannst du in der nächsten Stunde verbessern?}}
						\begin{center}
							%\vspace{1.1cm}
							\noindent\rule{12cm}{0.2pt}
							\vspace{1.1cm}
							\noindent\rule{12cm}{0.1pt}
							\vspace{1.1cm}
							\noindent\rule{12cm}{0.1pt}
							\vspace{1.1cm}
							\noindent\rule{12cm}{0.1pt}
							\vspace{1.1cm}
							\noindent\rule{12cm}{0.1pt}
							\vspace{1.1cm}
							\noindent\rule{12cm}{0.1pt}
							\vspace{1.1cm}
							\noindent\rule{12cm}{0.1pt}
							\vspace{1.1cm}
							\noindent\rule{12cm}{0.1pt}
						\end{center}
					\end{tcolorbox}
				\end{center}
								
				% *******************************************************************************
				% ***** end copy here ***********************************************************


\newpage
	\part{Additum}
		\section{Mehrwertige Alkohole}

			\textit{Die Gruppe der Alkohole ist sehr groß. Wie unterscheiden sich nun die verschiedenen Arten voneinander? Finde es mit den Experimenten heraus, die du schon oft durchgeführt hast.} \newline
			
			\begin{minipage}{0.7\textwidth}
				\noindent \textbf{Am Ende dieses Kapitels solltest du ... :}
				\begin{enumerate}
				    \item ... begründete Voraussagen zur Struktur von Teilchen bei Kenntnis der Eigenschaften der Alkanole zu machen.
				    \item ... Glycerin und Glykol als Beispiele für Polyalkohole zu nennen und deren Strukturformen aufzuzeichnen.
				\end{enumerate}	
				
				\noindent \textbf{Vorgehensweise:}
				\begin{enumerate}
					\item Arbeite in Gruppen von 4-5 SuS.
					\item Bearbeite die Arbeitsaufträge.
				\end{enumerate}	
			\end{minipage}
			\hspace{0.1\textwidth}
			\begin{minipage}{0.2\textwidth}
				\begin{tcolorbox}
					[enhanced,
					width=0.9\textwidth,
					colback=white,
					colframe=black,
					fonttitle=\sffamily\bfseries\large, 
					title=Zeit,  % search keyword::Zeit
					attach boxed title to top center={xshift=-0.0mm,yshift=-0.50mm},
					boxed title style={skin=enhancedfirst jigsaw,size=small,arc=1mm,bottom=-1mm,colframe=black,height=0.75cm},
					colbacktitle=black,
					drop lifted shadow]
					\centering
					\includegraphics[width=0.9\textwidth]{symbols/symbol_tex_time}
					
					\begin{center}
						\textbf{90min}
					\end{center}
				\end{tcolorbox}
			\end{minipage}
			
			\vspace{0.3cm}
			\begin{tcolorbox}[enhanced,
				colback=white,
				colframe=orange!60!red,
				fonttitle=\sffamily\bfseries\large, 
				title=Wiederholung,  % search keyword::Wiederholung
				attach boxed title to top left={xshift=3.2mm,yshift=-0.50mm},
				boxed title style={skin=enhancedfirst jigsaw,size=small,arc=1mm,bottom=-1mm,colframe=orange!60!red,height=0.75cm},
				colbacktitle=orange!60!red,
				% sharp corners,
				drop lifted shadow]	
				\begin{wrapfigure}{L}{0.15\textwidth}  
					\centering
					\vspace{-14pt}  % to align image with first line of text
					\includegraphics[width=0.9\textwidth]{symbols/symbol_tex_review}
				\end{wrapfigure}
				
				Schau dir noch einmal die Protokoll zu den anderen Experimenten an, mit denen du Stoffeigenschaften untersucht hast. Wie wurden sie durchgeführt und was konntest du daraus ableiten? 
				\vspace{1.1cm}
			\end{tcolorbox}
			
			\begin{center}
				\noindent\rule{18cm}{0.1pt}
			\end{center}
						
\newpage
			\subsection{Die experimentelle Untersuchung von mehrwertigen Alkoholen}
			
				\noindent \textbf{Auftrag: Vergleiche experimentell die Eigenschaften von einwertigen und mehrwertigen Alkoholen.}
				\begin{enumerate}
	            \item \textbf{Ausführliches Protokoll!}
	            	\begin{itemize}
	            		\item Auftrag, begründete Hypothesen, Material, Beobachtung, Auswertung
	                	\item Jede(r) schreibt ein eigenes Protokoll!
	                	\item Mögliche Rechercheaufgaben können im Unterricht erfolgen, wenn noch Zeit ist!
					\end{itemize}	            
	            \item \textbf{Hinweise zur Planung/Durchführung:}
	                \begin{itemize}
	                	\item Um sinnvoll zu vergleichen, solltest du auch die Eigenschaften des einwertigen Alkohols untersuchen.
	                	\item Achte darauf, dass die C-Kette identisch ist.
					\end{itemize}	            
	            \item  \textbf{Recherche für die Beobachtung: Nicht alle Eigenschaften können wir in der Schule experimentell untersuchen. Recherchiere daher folgende Eigenschaften für die einwertigen und mehrwertigen Alkohole (nimm 3 Vertreter z.B. C3-, C5-, C7-Kette):}
	                \begin{itemize}
	                	\item Schmelz- und Siedetemperaturen (in Diagramm)
	                	\item Flammtemperaturen (in Diagramm)
	                \end{itemize}
	            \item \textbf{Hinweise für die Auswertung:}
	            	\begin{itemize}
		                \item Achte bei der Auswertung auf den Zusammenhang von Struktur-Kräfte-Eigenschaften!
		                \item Werte auch die recherchierten Ergebnisse aus!
		                \item Nutze Strukturformeln um deine Beobachtungen und Auswertungen zu erläutern!
		                \item Fehleranalyse und Bezug zur Hypothese!
		                \item Quellenangaben!
	                \end{itemize}
	            \item \textbf{Hinweise zum Layout:}
	                \begin{itemize}
		                \item Alle Experimente in ein Protokoll, nicht ein Protokoll für jedes Experiment!
	                \end{itemize}
	            \item \textbf{Abgabe:}
	            	\begin{itemize}
		            	\item Das Protokoll wird bewertet!
	                	\item Deadline nach Absprache mit der Lehrkraft.
					\end{itemize}
				\end{enumerate}

% ***** start copy here *********************************************************
% *******************************************************************************
\vspace{4cm}	
			\begin{center}
				\begin{tcolorbox}[enhanced,
					width=0.75\textwidth,
					colback=white,
					colframe=blue,
					fonttitle=\sffamily\bfseries\large, 
					title=Verbindungen herstellen,  % search keyword::Informationstexte 
					attach boxed title to top left={xshift=3.2mm,yshift=-0.50mm},
					boxed title style={skin=enhancedfirst jigsaw,size=small,arc=1mm,bottom=-1mm,colframe=blue,height=0.75cm},
					colbacktitle=blue,
					drop lifted shadow]
					
					\begin{wrapfigure}{L}{0.15\textwidth}  
						\centering
						\vspace{-14pt}  % to align image with first line of text
						\includegraphics[width=0.7\textwidth]{symbols/symbol_tex_connect}
					\end{wrapfigure}
					
					\textit{Alle Kapitel sind miteinander verbunden. Wissen aus dem vorherigen Kapitel kann dir im nächsten Kapitel helften. Dazu musst du die Verbindungn zwischen den Kapiteln herstellen.} \newline
					
					\textbf{Auftrag: Nenne stichpunktartig drei wichtige Sachverhalte aus dem Kapitel, die du in Zukunft brauchen könntest.}
						
%					\begin{center}
%						%\vspace{1.1cm}
%						\noindent\rule{12cm}{0.2pt}
%						\vspace{1.1cm}
%						\noindent\rule{12cm}{0.1pt}
%						\vspace{1.1cm}
%						\noindent\rule{12cm}{0.1pt}
%						\vspace{1.1cm}
%						\noindent\rule{12cm}{0.1pt}
%						\vspace{1.1cm}
%						\noindent\rule{12cm}{0.1pt}
%						\vspace{0.1cm}
%					\end{center}

				\end{tcolorbox}
			\end{center}
			
% ***** end copy here *********************************************************
% *******************************************************************************

\newpage

				% ***** start copy here *********************************************************
				% *******************************************************************************
				%            __ _           _   _                        _                _   
				%  _ __ ___ / _| | ___  ___| |_(_) ___  _ __         ___| |__   __ _ _ __| |_ 
				% | '__/ _ \ |_| |/ _ \/ __| __| |/ _ \| '_ \ _____ / __| '_ \ / _` | '__| __|
				% | | |  __/  _| |  __/ (__| |_| | (_) | | | |_____| (__| | | | (_| | |  | |_ 
				% |_|  \___|_| |_|\___|\___|\__|_|\___/|_| |_|      \___|_| |_|\__,_|_|   \__|
				%                                                                            
				%
				
				\vspace{0.3cm}
				\begin{center}
					\begin{tcolorbox}[enhanced,
						width=0.75\textwidth,
						colback=white,
						colframe=darkgray,
						fonttitle=\sffamily\bfseries\large, 
						title=Kurzreflexion,  % search keyword::Informationstexte 
						attach boxed title to top left={xshift=3.2mm,yshift=-0.50mm},
						boxed title style={skin=enhancedfirst jigsaw,size=small,arc=1mm,bottom=-1mm,colframe=darkgray,height=0.75cm},
						colbacktitle=darkgray,
						drop lifted shadow]
						
						% \begin{figure}[htbp]  % not good in ecolorbox
						\textbf{Auftrag: Bewerte deine Arbeit in der letzten Einheit selbst.} \newline
						\begin{center}
							\begin{tikzpicture}[scale=1]
								\path (0:0cm) coordinate (O); % define coordinate for origin
								
								% draw the spiderweb
								\foreach \X in {1,...,\D}{
									\draw (\X*\A:0) -- (\X*\A:\R);
								}
								
								\foreach \Y in {0,...,\U}{
									\foreach \X in {1,...,\D}{
								\path (\X*\A:\Y*\R/\U) coordinate (D\X-\Y);
								\fill (D\X-\Y) circle (1pt);
									}
									\draw [opacity=0.3] (0:\Y*\R/\U) \foreach \X in {1,...,\D}{
										-- (\X*\A:\Y*\R/\U)
									} -- cycle;
								}
								
								% define labels for each dimension axis (names config option)
								\path (1*\A:\L) node (L1) {\tiny Verständnis};
								\path (2*\A:\L) node (L2) {\tiny alle Aufgaben gelöst};
								\path (3*\A:\L) node (L3) {\tiny Zeit effektiv genutz};
								\path (4*\A:\L) node (L4) {\tiny Einzelarbeit};
								\path (5*\A:\L) node (L5) {\tiny Gruppenarbeit};
							
							\end{tikzpicture}
							% \caption{Diagramm Slebstreflexion}
							% \caption{Spiderweb Diagram (\D~Dimensions, \U-Notch Scale, 3 Samples)}
							% \label{fig:spiderweb}
							% \end{figure} 
						\end{center}
						\textbf{{\Large Was kannst du in der nächsten Stunde verbessern?}}
						\begin{center}
							%\vspace{1.1cm}
							\noindent\rule{12cm}{0.2pt}
							\vspace{1.1cm}
							\noindent\rule{12cm}{0.1pt}
							\vspace{1.1cm}
							\noindent\rule{12cm}{0.1pt}
							\vspace{1.1cm}
							\noindent\rule{12cm}{0.1pt}
							\vspace{1.1cm}
							\noindent\rule{12cm}{0.1pt}
							\vspace{1.1cm}
							\noindent\rule{12cm}{0.1pt}
							\vspace{1.1cm}
							\noindent\rule{12cm}{0.1pt}
							\vspace{1.1cm}
							\noindent\rule{12cm}{0.1pt}
						\end{center}
					\end{tcolorbox}
				\end{center}
								
				% *******************************************************************************
				% ***** end copy here ***********************************************************

				
\newpage
		\section{Alkoholtest}
		
			\textit{Nachdem du herausgefunden hast, wie man Alkohol herstellt, passieren wieder mehr Unfälle, weil die Menschen zu viel trinken. Es gibt einen einfachen Test, den du noch an einigen verlassenen Tankstellen finden kannst, der zeigt, wie viel Alkohol eine Person getrunken hat. Der Test ist auch ein schönes Beispiel für angewandte Chemie! Man lernt dabei auch die Aldeyhden kennen!} \newline
				
			\begin{minipage}{0.7\textwidth}
				\noindent \textbf{Am Ende dieses Kapitels solltest du ... :}
				\begin{enumerate}
					\item ... das Reaktionsschema für die Oxidation der Alkohole zu Aldehyden oder Ketonen aufstellen können.
					\item ... den Begriff Oxidationszahl definieren können.
					\item ... die Regeln zum Aufstellen der Oxidationszahlen nennen können.
					\item ... die Begriffe Redoxreaktion, Oxidation und Reduktion mit Hilfe der Oxidationszahlen erläutern können. 
					\item ... die Strukturformeln der funktionellen Gruppen der Aldeyde und Ketone aufzeichnen können.
					\item ... die Verwendung ausgewählter Aldehyde und Ketone bzw. deren Vorkommen in der Natur zu erläutern.  
				\end{enumerate}	
				
				\noindent \textbf{Vorgehensweise:}
				\begin{enumerate}
					\item Arbeite in Gruppen von 4-5 SuS.
					\item Teilt euch die vier Texte auf!
					\item Lest jeweils euren eigenen Text. 
					\item Bearbeite die Arbeitsaufträge. 
					\item Tauscht eure Ergebnisse aus und löst die Aufträge gemeinsam. Ihr müsste euch gegenseitig austauschen!
				\end{enumerate}	
			\end{minipage}
			\hspace{0.1\textwidth}
			\begin{minipage}{0.2\textwidth}
				\begin{tcolorbox}
					[enhanced,
					width=0.9\textwidth,
					colback=white,
					colframe=black,
					fonttitle=\sffamily\bfseries\large, 
					title=Zeit,  % search keyword::Zeit
					attach boxed title to top center={xshift=-0.0mm,yshift=-0.50mm},
					boxed title style={skin=enhancedfirst jigsaw,size=small,arc=1mm,bottom=-1mm,colframe=black,height=0.75cm},
					colbacktitle=black,
					drop lifted shadow]
					\centering
					\includegraphics[width=0.9\textwidth]{symbols/symbol_tex_time}
					
					\begin{center}
						\textbf{90min}
					\end{center}
				\end{tcolorbox}
			\end{minipage}
			
			\begin{center}
				\noindent\rule{18cm}{0.1pt}
			\end{center}
						
\newpage	

		\subsection{Die chemischen Hintergründe des Alkoholtests}

			\noindent \textbf{Auftrag – Gruppe 1: Erkläre wie ein einfacher Alkoholtest funktioniert.}
			\begin{enumerate}
            	\item Stelle (mit deinen drei PartnerInnen) die \textbf{Reaktionsgleichung} für den Alkoholtest auf! (Tipp: Erst Wortgleichung, dann Reaktionsgleichung!!) 
            	\item Konzentriert euch auf das \textbf{Chrom-Teilchen} in den nicht-organischen Substanzen!
            	\item Verwende die geeigneten \textbf{LEWIS-Formeln für die organischen Verbindungen}!
            	\item \textbf{Erläutere} den Begriff \textit{Redoxreaktion} und \textit{Oxidationszahlen} daran!
            \end{enumerate}
	        
	        \begin{tcolorbox}[enhanced,
				colback=white,
				colframe=darkgray,
				fonttitle=\sffamily\bfseries\large, 
				title=Informationstext - Gruppe 1,  % search keyword::Informationstexte 
				attach boxed title to top left={xshift=3.2mm,yshift=-0.50mm},
				boxed title style={skin=enhancedfirst jigsaw,size=small,arc=1mm,bottom=-1mm,colframe=darkgray,height=0.75cm},
				colbacktitle=darkgray,
				drop lifted shadow]
				\begin{wrapfigure}{L}{0.15\textwidth}  
					\centering
					\vspace{-14pt}  % to align image with first line of text
					\includegraphics[width=0.9\textwidth]{symbols/symbol_tex_content}
				\end{wrapfigure}
				
				Um den Blutalkohol zu messen, gibt es mehrere Verfahren. Ein sehr einfaches Verfahren ist das \textit{Pusten}. Dabei nutzt die Polizei heute elektronische Meßgeräte. Man kann aber auch einfache Testgeräte an Tankstellen kaufen. Diese funktionieren über chemische Reaktionen.
				Dabei wird gelbes Kaliumdichromat (\ch{K2Cr2O7}) durch den Alkohol in der Atemluft zu grünem Chrom(III)sulfat reduziert. Gleichzeitig wird Ethanol zu Acetaldehyd (und dann zu Essigsäure) oxidiert. Diese Reaktion läuft nur in der Gegenwart von Schwefelsäure ab, die in geringen Mengen in dem Testgerät enthalten ist. 				
			\end{tcolorbox}
			
			\vspace{0.3cm}
			\noindent \textbf{Auftrag – Gruppe 2: Erkläre wie ein einfacher Alkoholtest funktioniert.}
			\begin{enumerate}
            	\item Stelle (mit deinen drei PartnerInnen) die \textbf{Reaktionsgleichung} für den Alkoholtest auf! (Tipp: Erst Wortgleichung, dann Reaktionsgleichung!!) 
            	\item Konzentriert euch auf das \textbf{Chrom-Teilchen} in den nicht-organischen Substanzen!
            	\item Verwende die geeigneten \textbf{LEWIS-Formeln für die organischen Verbindungen}!
            	\item \textbf{Erläutere} den Begriff \textit{Redoxreaktion} und \textit{Oxidationszahlen} daran!
   			\end{enumerate}
			
			\begin{tcolorbox}[enhanced,
				colback=white,
				colframe=darkgray,
				fonttitle=\sffamily\bfseries\large, 
				title=Informationstext - Gruppe 2,  % search keyword::Informationstexte 
				attach boxed title to top left={xshift=3.2mm,yshift=-0.50mm},
				boxed title style={skin=enhancedfirst jigsaw,size=small,arc=1mm,bottom=-1mm,colframe=darkgray,height=0.75cm},
				colbacktitle=darkgray,
				drop lifted shadow]
				\begin{wrapfigure}{L}{0.15\textwidth}  
					\centering
					\vspace{-14pt}  % to align image with first line of text
					\includegraphics[width=0.9\textwidth]{symbols/symbol_tex_content}
				\end{wrapfigure}
				
				Eine Redoxreaktion ist eine Reaktion bei der Oxidation und Reduktion parallel ablaufen. Bisher waren beide Begriffe mit dem Übergang von Sauerstoffatomen verbunden. Bei der Oxidation verbindet sich ein Element mit Sauerstoff, bei der Reduktion wird einer Verbindung Sauerstoff entzogen. 
				Schaut man sich die Sache genauer an, dann kann man Redoxreaktionen aber auch über Elektronenübergänge beschreiben. Bei der Reduktion werden Elektronen aufgenommen (die Oxidationszahl sinkt), bei der Oxidation werden Elektronen abgegeben (die Oxidationszahl steigt).			
			\end{tcolorbox}
			
			\vspace{0.3cm}
			\noindent \textbf{Auftrag – Gruppe 3: Erkläre wie ein einfacher Alkoholtest funktioniert.}
			\begin{enumerate}
            	\item Stelle (mit deinen drei PartnerInnen) die \textbf{Reaktionsgleichung} für den Alkoholtest auf! (Tipp: Erst Wortgleichung, dann Reaktionsgleichung!!) 
            	\item Konzentriert euch auf das \textbf{Chrom-Teilchen} in den nicht-organischen Substanzen!
            	\item Verwende die geeigneten \textbf{LEWIS-Formeln für die organischen Verbindungen}!
            	\item \textbf{Erläutere} den Begriff \textit{Redoxreaktion} und \textit{Oxidationszahlen} daran!
   			\end{enumerate}
   			
   			\begin{tcolorbox}[enhanced,
				colback=white,
				colframe=darkgray,
				fonttitle=\sffamily\bfseries\large, 
				title=Informationstext - Gruppe 3,  % search keyword::Informationstexte 
				attach boxed title to top left={xshift=3.2mm,yshift=-0.50mm},
				boxed title style={skin=enhancedfirst jigsaw,size=small,arc=1mm,bottom=-1mm,colframe=darkgray,height=0.75cm},
				colbacktitle=darkgray,
				drop lifted shadow]
				\begin{wrapfigure}{L}{0.15\textwidth}  
					\centering
					\vspace{-14pt}  % to align image with first line of text
					\includegraphics[width=0.9\textwidth]{symbols/symbol_tex_content}
				\end{wrapfigure}
				
				Oxidationszahlen sind hypothetische Ladungen, die man einem Atom in einer Verbindung gibt. Die Oxidationszahl kann negativ, Null oder positiv sein. Ein Element kann (in einer Verbindung) auch verschieden Oxidationszahlen haben. 
				Dabei gibt es einige Regeln zu beachten:
				\begin{itemize}
					\item Elemente (also Stoffe, die keine Verbindungen sind) haben immer die Oxidationszahl 0.
					\item Wasserstoff hat immer die Oxidationszahl +I. 
					\item Sauerstoff hat  meistens die Oxidationszahl -II. 
					\item Ionen haben die gleiche Oxidationszahl wie ihre Ladung. 
				\end{itemize}
				Bei organischen Verbindungen ergibt sich die Oxidationszahl des jeweiligen Kohlenstoffatoms aus den damit verbundenen Atomen bzw. funktionellen Gruppen. Jedes Kohlenstoffatom wird einzeln betrachtet.
				\begin{center}
				\includegraphics[width=0.5\textwidth]{images/oxidationnumbers_lewis}
				\end{center}		
			\end{tcolorbox}
   			
   			\vspace{0.3cm}
			\noindent \textbf{Auftrag – Gruppe 4: Erkläre wie ein einfacher Alkoholtest funktioniert.}
			\begin{enumerate}
            	\item Stelle (mit deinen drei PartnerInnen) die \textbf{Reaktionsgleichung} für den Alkoholtest auf! (Tipp: Erst Wortgleichung, dann Reaktionsgleichung!!) 
            	\item Konzentriert euch auf das \textbf{Chrom-Teilchen} in den nicht-organischen Substanzen!
            	\item Verwende die geeigneten \textbf{LEWIS-Formeln für die organischen Verbindungen}!
            	\item \textbf{Erläutere} den Begriff \textit{Redoxreaktion} und \textit{Oxidationszahlen} daran!
   			\end{enumerate}   	
		
			\begin{tcolorbox}[enhanced,
				colback=white,
				colframe=darkgray,
				fonttitle=\sffamily\bfseries\large, 
				title=Informationstext - Gruppe 4,  % search keyword::Informationstexte 
				attach boxed title to top left={xshift=3.2mm,yshift=-0.50mm},
				boxed title style={skin=enhancedfirst jigsaw,size=small,arc=1mm,bottom=-1mm,colframe=darkgray,height=0.75cm},
				colbacktitle=darkgray,
				drop lifted shadow]
				\begin{wrapfigure}{L}{0.15\textwidth}  
					\centering
					\vspace{-14pt}  % to align image with first line of text
					\includegraphics[width=0.9\textwidth]{symbols/symbol_tex_content}
				\end{wrapfigure}
				
				Aldehyde sind „dehydrierte“ Alkohole. Sie haben die funktionelle Gruppe -CHO. Diese Gruppe wird auch als Aldehydgruppe bezeichnet. Kohlenstoff hat hier die Oxidationszahl +1. Aldehyde werden auch als Alkanale bezeichnet.
				\begin{center}
				\includegraphics[width=0.125\textwidth]{images/FunktionelleGruppen_Aldehyde.svg}
				\end{center}
				Die Benennung der Aldehyde leitet sich aus der Nomenklatur der Alkane ab. Die Aldehyde haben die Endung -al; also Methanal (auch bekannt als Formaldehy), Ethanal (auch bekannt als Acetaldehy), Propanal etc. \newline
				Die C=O Bindung ist stark polar. Es kann also wie bei den Alkohlen zur Ausbildung von Dipol-Dipol-Kräften (bzw. Wasserstoffbrücken) kommen. Die Stoffe sind also bis zu einem bestimmten Punkt wasserlöslich.		
			\end{tcolorbox}		

% ***** start copy here *********************************************************
% *******************************************************************************
\vspace{12cm}	
			\begin{center}
				\begin{tcolorbox}[enhanced,
					width=0.75\textwidth,
					colback=white,
					colframe=blue,
					fonttitle=\sffamily\bfseries\large, 
					title=Verbindungen herstellen,  % search keyword::Informationstexte 
					attach boxed title to top left={xshift=3.2mm,yshift=-0.50mm},
					boxed title style={skin=enhancedfirst jigsaw,size=small,arc=1mm,bottom=-1mm,colframe=blue,height=0.75cm},
					colbacktitle=blue,
					drop lifted shadow]
					
					\begin{wrapfigure}{L}{0.15\textwidth}  
						\centering
						\vspace{-14pt}  % to align image with first line of text
						\includegraphics[width=0.7\textwidth]{symbols/symbol_tex_connect}
					\end{wrapfigure}
					
					\textit{Alle Kapitel sind miteinander verbunden. Wissen aus dem vorherigen Kapitel kann dir im nächsten Kapitel helften. Dazu musst du die Verbindungn zwischen den Kapiteln herstellen.} \newline
					
					\textbf{Auftrag: Nenne stichpunktartig drei wichtige Sachverhalte aus dem Kapitel, die du in Zukunft brauchen könntest.}
						
%					\begin{center}
%						%\vspace{1.1cm}
%						\noindent\rule{12cm}{0.2pt}
%						\vspace{1.1cm}
%						\noindent\rule{12cm}{0.1pt}
%						\vspace{1.1cm}
%						\noindent\rule{12cm}{0.1pt}
%						\vspace{1.1cm}
%						\noindent\rule{12cm}{0.1pt}
%						\vspace{1.1cm}
%						\noindent\rule{12cm}{0.1pt}
%						\vspace{0.1cm}
%					\end{center}

				\end{tcolorbox}
			\end{center}
			
% ***** end copy here *********************************************************
% *******************************************************************************			

\newpage

	% ***** start copy here *********************************************************
	% *******************************************************************************
	%            __ _           _   _                        _                _   
	%  _ __ ___ / _| | ___  ___| |_(_) ___  _ __         ___| |__   __ _ _ __| |_ 
	% | '__/ _ \ |_| |/ _ \/ __| __| |/ _ \| '_ \ _____ / __| '_ \ / _` | '__| __|
	% | | |  __/  _| |  __/ (__| |_| | (_) | | | |_____| (__| | | | (_| | |  | |_ 
	% |_|  \___|_| |_|\___|\___|\__|_|\___/|_| |_|      \___|_| |_|\__,_|_|   \__|
	%                                                                            
	%
	
	\vspace{0.3cm}
	\begin{center}
		\begin{tcolorbox}[enhanced,
			width=0.75\textwidth,
			colback=white,
			colframe=black,
			fonttitle=\sffamily\bfseries\large, 
			title=Kurzreflexion,  % search keyword::Informationstexte 
			attach boxed title to top left={xshift=3.2mm,yshift=-0.50mm},
			boxed title style={skin=enhancedfirst jigsaw,size=small,arc=1mm,bottom=-1mm,colframe=darkgray,height=0.75cm},
			colbacktitle=black,
			drop lifted shadow]
			
			% \begin{figure}[htbp]  % not good in ecolorbox
			\textbf{Auftrag: Bewerte deine Arbeit in der letzten Einheit selbst.} \newline
			\begin{center}
				\begin{tikzpicture}[scale=1]
					\path (0:0cm) coordinate (O); % define coordinate for origin
					
					% draw the spiderweb
					\foreach \X in {1,...,\D}{
						\draw (\X*\A:0) -- (\X*\A:\R);
					}
					
					\foreach \Y in {0,...,\U}{
						\foreach \X in {1,...,\D}{
					\path (\X*\A:\Y*\R/\U) coordinate (D\X-\Y);
					\fill (D\X-\Y) circle (1pt);
						}
						\draw [opacity=0.3] (0:\Y*\R/\U) \foreach \X in {1,...,\D}{
							-- (\X*\A:\Y*\R/\U)
						} -- cycle;
					}
					
					% define labels for each dimension axis (names config option)
					\path (1*\A:\L) node (L1) {\tiny Verständnis};
					\path (2*\A:\L) node (L2) {\tiny alle Aufgaben gelöst};
					\path (3*\A:\L) node (L3) {\tiny Zeit effektiv genutz};
					\path (4*\A:\L) node (L4) {\tiny Einzelarbeit};
					\path (5*\A:\L) node (L5) {\tiny Gruppenarbeit};
				
				\end{tikzpicture}
				% \caption{Diagramm Slebstreflexion}
				% \caption{Spiderweb Diagram (\D~Dimensions, \U-Notch Scale, 3 Samples)}
				% \label{fig:spiderweb}
				% \end{figure} 
			\end{center}
			\textbf{{\Large Was könntest du in der nächsten Stunde verbessern?}}
			\begin{center}
				%\vspace{1.1cm}
				\noindent\rule{12cm}{0.2pt}
				\vspace{1.1cm}
				\noindent\rule{12cm}{0.1pt}
				\vspace{1.1cm}
				\noindent\rule{12cm}{0.1pt}
				\vspace{1.1cm}
				\noindent\rule{12cm}{0.1pt}
				\vspace{1.1cm}
				\noindent\rule{12cm}{0.1pt}
				\vspace{1.1cm}
				\noindent\rule{12cm}{0.1pt}
				\vspace{1.1cm}
				\noindent\rule{12cm}{0.1pt}
				\vspace{1.1cm}
				\noindent\rule{12cm}{0.1pt}
			\end{center}
		\end{tcolorbox}
	\end{center}
					
	% *******************************************************************************
	% ***** end copy here ***********************************************************
	




\newpage



		\section{Aldehydnachweis}
%TODO: Zombieland -> Medizin Zuckernachweis Diabetes		
			\textit{In Zombieland sind viele Sachen kaputt. Auch die Weihnachtskugeln für den Weihnachtsbaum. Wäre doch schade drum. Und einen ordentlichen Spiegel könnte man auch mal wieder gebrauchen. Zufälligerweise kann der Nachweis der Aldehyde ein wenig aushelfen.}
			
			\begin{minipage}{0.7\textwidth}
				\noindent \textbf{Am Ende dieses Kapitels solltest du ... :}
				\begin{enumerate}
					\item ... die Aldehyde mit der Silberspiegel-Probe nachweisen können.
					\item ... den Nachweis mit Hilfe einer Redoxreaktion erläutern können. 
				\end{enumerate}	
				
				\noindent \textbf{Vorgehensweise:}
				\begin{enumerate}
					\item Arbeite in Gruppen von 4-5 SuS.
					\item Bearbeite die Arbeitsaufträge.
				\end{enumerate}	
			\end{minipage}
			\hspace{0.1\textwidth}
			\begin{minipage}{0.2\textwidth}
				\begin{tcolorbox}
					[enhanced,
					width=0.9\textwidth,
					colback=white,
					colframe=black,
					fonttitle=\sffamily\bfseries\large, 
					title=Zeit,  % search keyword::Zeit
					attach boxed title to top center={xshift=-0.0mm,yshift=-0.50mm},
					boxed title style={skin=enhancedfirst jigsaw,size=small,arc=1mm,bottom=-1mm,colframe=black,height=0.75cm},
					colbacktitle=black,
					drop lifted shadow]
					\centering
					\includegraphics[width=0.9\textwidth]{symbols/symbol_tex_time}
					
					\begin{center}
						\textbf{90min}
					\end{center}
				\end{tcolorbox}
			\end{minipage}
			
			\begin{center}
				\noindent\rule{18cm}{0.1pt}
			\end{center}
\newpage
		
		\subsection{Die Silberspiegelprobe}
	
			\noindent \textbf{Auftrag: Weise Aldehyde mit Hilfe der Silberspiegel-Probe nach.}
			\begin{enumerate}
				\item Lies die Durchführung.
				\item Führe das Experiment durch und protokolliere es.
				\item Beantworte die gegebenen Fragen als Teil der Auswertung.
			\end{enumerate}
			
			\begin{tcolorbox}[enhanced,
				colback=white,
				colframe=green!30!black,
				fonttitle=\sffamily\bfseries\large, 
				title=Durchführung,  % search keyword::Durchführung
				attach boxed title to top left={xshift=3.2mm,yshift=-0.50mm},
				boxed title style={skin=enhancedfirst jigsaw,size=small,arc=1mm,bottom=-1mm,colframe=green!50!black,height=0.75cm},
				colbacktitle=green!50!black,
				drop lifted shadow]
				\begin{wrapfigure}{L}{0.15\textwidth}  
					\centering
					\vspace{-14pt}  % to align image with first line of text
					\includegraphics[width=0.7\textwidth]{symbols/symbol_tex_method}
				\end{wrapfigure}
				
					In je ein Reagenzglas werden 5 ml Silbernitrat-Lösung (w = 5 \%) gegeben. In diese Lösungen wird vorsichtig so viel der Ammoniak-Lösung (w = 10 \%) getropft, bis sich der Niederschlag, der sich gebildet hat, gerade wieder auflöst. Hiernach werden 3 ml Glucose- bzw. Fructose-Lösung hinzugegeben (beide Stoffe sind auch Aldehyde) und vorsichtig geschüttelt. Anschließend stellt man die Reagenzgläser in ein Becherglas mit heißem Wasser. \footnote{Quelle(angepasst): https://www.chemieunterricht.de/dc2/haus/v021.htm [Stand: 28.7.2020]}
					
					
				
			\end{tcolorbox}
			
			\vspace{0.3cm}
			\noindent \textbf{Fragen für die Auswertung:}
			\begin{enumerate}
				\item Die Silberspiegel-Probe hat auch einen anderen Namen. \textbf{Nenne} ihn.
				\item \textbf{Zeichne} die LEWIS-Strukturformel für Glucose oder Fructose. Achte dabei darauf, dass du die \textit{FISCHER-Projektion} nutzt.
				\item \textbf{Erkläre} deine Beobachtung.
				\item \textbf{Recherchiere} und \textbf{erläutere}, warum es sich bei der Silberspiegel-Probe um eine Redox-Reaktion handelt.
			\end{enumerate}
			
			\begin{tcolorbox}[enhanced,
				colback=white,
				colframe=black,
				fonttitle=\sffamily\bfseries\large, 
				title=Internet-Quelle (URL),  % search keyword::URL
				attach boxed title to top left={xshift=3.2mm,yshift=-0.50mm},
				boxed title style={skin=enhancedfirst jigsaw,size=small,arc=1mm,bottom=-1mm,colframe=black,height=0.75cm},
				colbacktitle=black,
				drop lifted shadow]
				\begin{wrapfigure}{L}{0.15\textwidth}  
					\centering
					\vspace{-14pt}  % to align image with first line of text
					\includegraphics[width=0.9\textwidth]{images/maus_weihnachtskugel}
				\end{wrapfigure}
				
					Du kannst dir hier anschauen, wie Weihnachtskugeln gemacht werden!!\footnote{Wenn du den QR-Code nicht scannen kannst, kannst du auch direkt aus der PDF-Datei auf die URL klicken}. \newline
					\textbf{Quelle} [Stand:29.7.2020]: \newline 
					\url{https://kinder.wdr.de/tv/die-sendung-mit-der-maus/av/video-sachgeschichte--christbaumkugel-100.html}
				\vspace{0.5cm}  % to fill empty space in tcolorbox
			\end{tcolorbox}
			
% ***** start copy here *********************************************************
% *******************************************************************************
\vspace{1cm}	
			\begin{center}
				\begin{tcolorbox}[enhanced,
					width=0.75\textwidth,
					colback=white,
					colframe=blue,
					fonttitle=\sffamily\bfseries\large, 
					title=Verbindungen herstellen,  % search keyword::Informationstexte 
					attach boxed title to top left={xshift=3.2mm,yshift=-0.50mm},
					boxed title style={skin=enhancedfirst jigsaw,size=small,arc=1mm,bottom=-1mm,colframe=blue,height=0.75cm},
					colbacktitle=blue,
					drop lifted shadow]
					
					\begin{wrapfigure}{L}{0.15\textwidth}  
						\centering
						\vspace{-14pt}  % to align image with first line of text
						\includegraphics[width=0.7\textwidth]{symbols/symbol_tex_connect}
					\end{wrapfigure}
					
					\textit{Alle Kapitel sind miteinander verbunden. Wissen aus dem vorherigen Kapitel kann dir im nächsten Kapitel helften. Dazu musst du die Verbindungn zwischen den Kapiteln herstellen.} \newline
					
					\textbf{Auftrag: Nenne stichpunktartig drei wichtige Sachverhalte aus dem Kapitel, die du in Zukunft brauchen könntest.}
						
%					\begin{center}
%						%\vspace{1.1cm}
%						\noindent\rule{12cm}{0.2pt}
%						\vspace{1.1cm}
%						\noindent\rule{12cm}{0.1pt}
%						\vspace{1.1cm}
%						\noindent\rule{12cm}{0.1pt}
%						\vspace{1.1cm}
%						\noindent\rule{12cm}{0.1pt}
%						\vspace{1.1cm}
%						\noindent\rule{12cm}{0.1pt}
%						\vspace{0.1cm}
%					\end{center}

				\end{tcolorbox}
			\end{center}
			
% ***** end copy here *********************************************************
% *******************************************************************************

\newpage

				% ***** start copy here *********************************************************
				% *******************************************************************************
				%            __ _           _   _                        _                _   
				%  _ __ ___ / _| | ___  ___| |_(_) ___  _ __         ___| |__   __ _ _ __| |_ 
				% | '__/ _ \ |_| |/ _ \/ __| __| |/ _ \| '_ \ _____ / __| '_ \ / _` | '__| __|
				% | | |  __/  _| |  __/ (__| |_| | (_) | | | |_____| (__| | | | (_| | |  | |_ 
				% |_|  \___|_| |_|\___|\___|\__|_|\___/|_| |_|      \___|_| |_|\__,_|_|   \__|
				%                                                                            
				%
				
				\vspace{0.3cm}
				\begin{center}
					\begin{tcolorbox}[enhanced,
						width=0.75\textwidth,
						colback=white,
						colframe=darkgray,
						fonttitle=\sffamily\bfseries\large, 
						title=Kurzreflexion,  % search keyword::Informationstexte 
						attach boxed title to top left={xshift=3.2mm,yshift=-0.50mm},
						boxed title style={skin=enhancedfirst jigsaw,size=small,arc=1mm,bottom=-1mm,colframe=darkgray,height=0.75cm},
						colbacktitle=darkgray,
						drop lifted shadow]
						
						% \begin{figure}[htbp]  % not good in ecolorbox
						\textbf{Auftrag: Bewerte deine Arbeit in der letzten Einheit selbst.} \newline
						\begin{center}
							\begin{tikzpicture}[scale=1]
								\path (0:0cm) coordinate (O); % define coordinate for origin
								
								% draw the spiderweb
								\foreach \X in {1,...,\D}{
									\draw (\X*\A:0) -- (\X*\A:\R);
								}
								
								\foreach \Y in {0,...,\U}{
									\foreach \X in {1,...,\D}{
								\path (\X*\A:\Y*\R/\U) coordinate (D\X-\Y);
								\fill (D\X-\Y) circle (1pt);
									}
									\draw [opacity=0.3] (0:\Y*\R/\U) \foreach \X in {1,...,\D}{
										-- (\X*\A:\Y*\R/\U)
									} -- cycle;
								}
								
								% define labels for each dimension axis (names config option)
								\path (1*\A:\L) node (L1) {\tiny Verständnis};
								\path (2*\A:\L) node (L2) {\tiny alle Aufgaben gelöst};
								\path (3*\A:\L) node (L3) {\tiny Zeit effektiv genutz};
								\path (4*\A:\L) node (L4) {\tiny Einzelarbeit};
								\path (5*\A:\L) node (L5) {\tiny Gruppenarbeit};
							
							\end{tikzpicture}
							% \caption{Diagramm Slebstreflexion}
							% \caption{Spiderweb Diagram (\D~Dimensions, \U-Notch Scale, 3 Samples)}
							% \label{fig:spiderweb}
							% \end{figure} 
						\end{center}
						\textbf{{\Large Was kannst du in der nächsten Stunde verbessern?}}
						\begin{center}
							%\vspace{1.1cm}
							\noindent\rule{12cm}{0.2pt}
							\vspace{1.1cm}
							\noindent\rule{12cm}{0.1pt}
							\vspace{1.1cm}
							\noindent\rule{12cm}{0.1pt}
							\vspace{1.1cm}
							\noindent\rule{12cm}{0.1pt}
							\vspace{1.1cm}
							\noindent\rule{12cm}{0.1pt}
							\vspace{1.1cm}
							\noindent\rule{12cm}{0.1pt}
							\vspace{1.1cm}
							\noindent\rule{12cm}{0.1pt}
							\vspace{1.1cm}
							\noindent\rule{12cm}{0.1pt}
						\end{center}
					\end{tcolorbox}
				\end{center}
								
				% *******************************************************************************
				% ***** end copy here ***********************************************************			
			
\end{document}

